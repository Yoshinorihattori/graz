%https://medemanabu.net/latex/documentclass/
\documentclass[12pt,oneside]{jbook}
%\documentclass[a4j,12pt,oneside]{jbook}
\usepackage{master_thesis2020}
\usepackage{comment}
\usepackage{amssymb}

\usepackage{booktabs}
\usepackage{siunitx}
\usepackage{breqn}%数式を改行する


%https://www.overleaf.com/learn/latex/Nomenclatures
\usepackage{nomencl}
\makenomenclature
\usepackage{etoolbox}
\renewcommand\nomgroup[1]{%
  \item[\bfseries
  \ifstrequal{#1}{A}{Physics Constants}{%
  \ifstrequal{#1}{B}{Material Properties}{%
  \ifstrequal{#1}{C}{Other Symbols}{}}}%
]}
\newcommand{\nomunit}[1]{%
\renewcommand{\nomentryend}{\hspace*{\fill}#1}}


\begin{document}
\thesistitle
	{Forced convective heat transfer in cylindrical pipe flows} % 論文題目
	{MD18060} % 学籍番号
	{Yoshinori}{Hattori} % 氏名
    {}{}
\tableofcontents

\mbox{}
\nomenclature[A, 02]{$c$}{Speed of light in a vacuum inertial system\nomunit{$299,792,458\, m/s$}}
\nomenclature[A, 03]{$h$}{Plank Constant\nomunit{$6.62607 \times 10^{-34}\, Js$}}
\nomenclature[B, 01]{$T$}{Temperature\nomunit{$K$}}
\nomenclature[B, 02]{$C_{p}$}{Specific heat capasity\nomunit{$J \cdot kg^{-1} K^{-1}$}}
\nomenclature[B, 03]{$\lambda$}{Thermal conductivity\nomunit{$W m^{-1} K^{-1}$}}
\nomenclature[B, 04]{$\mu$}{Dynamic viscosity\nomunit{$Pa \cdot s$}}
\nomenclature[B, 05]{$\rho$}{Density\nomunit{$kg/m^{3}$}}
\nomenclature[B, 06]{$Pr$}{Prandtl number\nomunit{-}}

\nomenclature[C]{$V$}{Constant Volume}
\nomenclature[C]{$\rho$}{Friction Index}
\printnomenclature


\chapter{Introduction}
\section{Study background}
Forced convective heat transfer lies at the heart of many aspect of cooling technology and it is therefore desirable to understand its properties as well as possible.
Effective cooling technology is constantly being required to wide variety of industrial engineering aplication.
To achieve effective coolant system requires comprefensive research of heat transfer coefficient with a wide variety of flow condition.
Although many reserchers have been focusing on experimental and computational research, heat transfer coefficient vary with Reynolds number is still unclear.
To this end, many reserchers have been focusing on heat transfer from experimental and computational research aspect.
However, heat transfer in transitional and turbulent flow is still very challenging task for both experimental and computational research.
\begin{enumerate}
	\item Experimental research\\

	\item Computational research\\
	Direct numerical simuration
	In technology, flows regime and heat transfer plays an important role in considerting engineering issues.
	Navie-Stokes equations describe the relation of variable flows.
	\begin{equation}
		\
	\end{equation}
	However, deterministic solution of the equations are only valid for small disturbances in the initial and boundary condition.
	In physically, it is hard to get initial and boundary conditions in infinite accurate.
	Turbulent has a large amount of fluctuations, i.e. turbulent is completely different kind of laminar flows.
	Direct Numerical Simuration (DNS) is one of the simulation way to predict flow forms.
	The object of the simuration is to solve the compelete set of equation of motion without using any model.
	From Kolmogorov lenght scale, total number of cumputations is derivered following equation (\ref{DNS_total_number_of_computation}).
	The DNS require large amount of total number of computations.
	\begin{equation}
	    \mathcal N \times \mathcal M= \mathcal O (Re^{11/4})
	    \label{DNS_total_number_of_computation}
	\end{equation}
	The equauation implies the limitation of the DNS and that is directly connected to computer technology.
	Normally, engineeres is interested in high Reynolds number such as aircraft or atmospheric boundary layer.
	However, such high Reynolds number requires huge amount of total number of computations and it's far from reality.
	Large eddy simulation
\end{enumerate}

One attempt to improve our understanding of entanglement is the study of our ability to perform experimental investigation


These coolant technology is used wide varaety of coolant applications such as electric devices, automotive, and plant factory.
Considering heat transfer issues, heat transfer coefficients are one of the most important numbers.
The Nusselt number (Nu) is a dimensionless number which represents the ratio of convective (h) and conductive heat transfer (k), as expressed in Equation.
\section{Previous research}

The equauation implies the limitation of the DNS and that is directly connected to computer technology.
Normally, engineeres is interested in high Reynolds number such as aircraft or atmospheric boundary layer.
However, such high Reynolds number requires huge amount of total number of computations and it's far from reality.

Therefore, it is nessesary to get experimental data for correlations of heat transfer and flow condition and the Reynolds number.

%attention!!!!!!!!!!!
Many studies have pointed out that a heat transfer coefficient varies depending on the type of flow: laminar, transition and turbulent.
Gnienlinski\cite{Gnielinski1975}\cite{Gnielinski1976} showed a calculation method about heat transfer coefficients for the laminar, transitional and turbulent flows.
Bertsche et al.\cite{Bertsche2016} focused on reliable prediction of the heat transfer coefficient for transitional flows.
In their study, they showed experimental the heat transfer coefficients for the Reynolds number, $500 < Re < 23000$, and the Prandtl number, $7 < Pr < 41$.\\
However, not so many data is available for experimental data of laminar-to-turbulent transitional region.
More studies should be conducted to obtain experimental data for high the Prandtl number and transitional flows.
In this study, the author focused on forced convective heat transfer in flow of water and glycole in a cylindrical pipe.
A 50/50vol\% mixture of water and glycole, which is a typical liquid coolant in automotive applications, was used as an operating fluid.
The experiment was carried out by considering a board range of Reynolds numbers, spanning from a laminar to fully turbulent flow.
Moreover, the measurements of the wall friction coefficients was also performed in this study.


\subsection{Skin friction coefficients}
The skin friction coefficients for laminar flow is descrived following equation.
\begin{equation}
    C_{f,lam}=\frac{16}{Re_{b}}
    \label{cf_laminar}
\end{equation}

Konakov\cite{Petukhov1970} showed the skin friction coefficients for turbulent flow.
\begin{equation}
	C_{f,turb}=0.25(1.8log(Re_{b})-1.64)^{-2}
	\label{cf_turbulent}
\end{equation}

\subsection{The heat transfer coefficients}
Gunienski \cite{Gnielinski2013} showed correlations for each flow conditions: laminar, transitional and turbulent, respectively.
Gunienski \cite{Gnielinski2013} showed calculation method for laminar flow.
\begin{equation}
    Nu_{lam}=(3.66^{3}+0.7^{3}+(1.615(Re_{b}Pr_{b}\frac{d_{i}}{L})^{1/3})^{3})^{1/3}
    \label{Nu_laminar}
\end{equation}
%The range 0.1<<Pr_{b}<<1000, 10^{4}<<Re_{b}<<10^{6}.
He showed calculation method for turbulent flow.
\begin{equation}
    Nu_{turb}=\frac{\frac{C_{f,turb}}{2Re\cdot Pr_{b}}}{1+12.7 \sqrt{\frac{C_{f,turb}}{2}}(Pr_{b}^{2/3}-1)}\cdot (\frac{Pr_{b}}{Pr_{w}})^{0.11}
    \label{Nu_turbulent}
%Nu_{turb}=\frac{\frac{C_{f}}{2\cdot Re\cdot Pr_{b}}{(1+12.7\frac{C_{f}}{2}}\cdot (Pr^{\frac{2}{3}}-1)}\cdot(1+(\frac{d_{h}}{l})^{\frac{2}{3}})\label{Nu_turblent}
\end{equation}
The range is
\begin{equation}
    0.1<<Pr_{b}<<1000, 10^{4}<<Re_{b}<<10^{6}
\end{equation}
He presented transitional flow as a liner interpolation between turbulent and laminar flow.
\begin{equation}
    Nu_{m}=(1-r)Nu_{m,lam}+rNu_{m,turb}
    \label{Nu_transitional}
\end{equation}
\begin{equation}
    r=\frac{Re_{b}-2300}{10^{4}-2300}
	\label{Nu_transitional2}
\end{equation}

\chapter{Methodology}
\section{Material properties}

\subsection{Parameters}
A 50/50vol\% mixture of water and glycole which is a typical liquid coolant in automotive applications were used as a operating fluid.
The material properties of the operationg fluid is temperature dependecy.
As bigger temperature difference between a pipe wall and the bulk, the material properties pronounced.


specific heat capacity
\begin{equation}
	c_{p} = A_{c_{p}}+B_{c_{p}} T = 2.0148 + 4.50E-3T
	\label{Cp}
\end{equation}
\begin{figure}[ht]
	\vspace{0zh}
	\begin{center}
		\includegraphics[width=0.65\linewidth]{fig/mp_cp.pdf}
		\vspace{-1zh}
		\caption{Specific heat capacisty vary with temperature}
		\label{cp}
	\end{center}
	\vspace{0zh}
\end{figure}


thermal condictivity
\begin{equation}
	\lambda = A_{\lambda}+B_{\lambda} T = 0.2134 + 6.071E-4T
	\label{lambda}
\end{equation}
\begin{figure}[ht]
	\vspace{0zh}
	\begin{center}
		\includegraphics[width=0.65\linewidth]{fig/mp_lambda.pdf}
		\vspace{-1zh}
		\caption{Thermal conductivity vary with temperature}
		\label{lambda}
	\end{center}
	\vspace{0zh}
\end{figure}


dynamic viscosity
\begin{equation}
	\mu = A_{\mu}\cdot \exp \left( \frac{B_{\mu}}{T+C_{\mu}} \right) = 1.1001E-4\exp \left( \frac{325.85}{T-207.30} \right)
	\label{mu}
\end{equation}
\begin{figure}[ht]
	\vspace{0zh}
	\begin{center}
		\includegraphics[width=0.65\linewidth]{fig/mp_mu.pdf}
		\vspace{-1zh}
		\caption{Dynamic viscosity vary with temperature}
		\label{mu}
	\end{center}
	\vspace{0zh}
\end{figure}


density
\begin{equation}
	\rho = A_{\rho}+B_{\rho} T = 1268.28 -0.66T
	\label{rho}
\end{equation}
\begin{figure}[ht]
    \vspace{0zh}
	\begin{center}
		\includegraphics[width=0.65\linewidth]{fig/mp_rho.pdf}
		\vspace{-1zh}
		\caption{Density vary with temperature}
		\label{rho}
	\end{center}
	\vspace{0zh}
\end{figure}


prandtl number
\begin{equation}
	Pr = \frac{\nu}{\alpha}= \frac{\mu \cdot c_{p}}{\lambda}
	\label{pr}
\end{equation}
\begin{figure}[ht]
	\vspace{0zh}
	\begin{center}
		\includegraphics[width=0.65\linewidth]{fig/mp_pr.pdf}
		\vspace{-1zh}
		\caption{Prandtl number vary with temperature}
		\label{pr}
	\end{center}
	\vspace{0zh}
\end{figure}

\subsection{Effect of viscosity and density}
DNS
Effect of viscosity 20\%
Effect og density   0.5\%
experimental

\clearpage

\subsection{Cooling and heating}
Bertsche et al.\cite{Bertsche2016} shows experimental investigation on heat transfer in cylindrical pipe flow by using a 50/50vol\% mixture of water and glycole with cooling.
The operationg fluid is the same as our study, however they measured the heat transfer by cooling the liquid.
For cooling the liquid, the temperature at the wall is lower than bulk temperature.
In other way, for heating the liquid, the temperature at the wall is higher that bulk temperature.
As shown in material properties, the viscosity show strong temperature dependence and the curve isn't linear function.
Thus, the heat transfer doesn't shows the same value with cooling and heating even though a Prandtl number is the same.
\begin{figure}[ht]
    \vspace{0zh}
	\begin{center}
		\includegraphics[width=1\linewidth]{fig/heating_and_cooling.pdf}
		\vspace{-2zh}
		\caption{Thermal boundary layer grows in horizontal axis for heating liquid(left) and cooling liquid(right).}
		\label{heating}
	\end{center}
	\vspace{0zh}
\end{figure}

\clearpage
\section{Hydro and thermal boundary layer}

\chapter{Experimental facilities}
\section{Experimental loop}
%Experimental Setup
Figure. \ref{loop} shows a diagram of the experimental loop.
The experimental basically loop consists of a heat exchanger, a pump, a Coriolis mass flow rate, a welder, a reservoir, and a test section.
The heat exchanger keeps a thermal stationary condition in the flow pipe.
The Coriolis mass flow rate is controlled by the pump and a bypass valve C, which is located in parallel to the pump.
The pipe is thermally insulated by using glass wool.\\

\begin{figure}[ht]
	\vspace{0zh}
	\begin{center}
		\includegraphics[width=1\linewidth]{fig/loop.pdf}
		\vspace{-3zh}
		\caption{Process flow diagram of the test facilities including test section.}
		\label{loop}
	\end{center}
	\vspace{0zh}
\end{figure}


\section{Test section}

\subsection{Thermal and velocity boundary layer}
Figure. \ref{testsection} shows the velocity and thermal boundary layer of developments vary with the horizontal axis in the test section.
%thermal boundary layer of WHAT???
The velocity and thermal profiles are shown in blue and red color, respectively.
The test section is made of stainless steel (1.4301) with an inner diameter $d_{i}=12[mm]$ and outer diameter $d_{o}=15[mm]$.
Highly accurate resistance thermall probes (PT-100) are used to find out the inlet and outlet bulk temperatures ($T_{in}$, $T_{out}$and wall temperature ($T_{w}$).
Moreover, thermocouples (Type-K) are used to measure temperature gradients in the flow direction.

The test section consists of entrance, heated and thermal equalized parts.
\begin{enumerate}
  \item Entrance part\\
  The first part of the test section is a $1.2\ [m]$ length entrance part, which is sufficiently long to ensure producing dynamically developed flow condition at the exist.
  The bulk temperature $T_{in}$ at this section was measured by PT-100.
  \item Heated part\\
  The second part of the test section is a $2.0\ [m]$ length heated part, which is sufficiently long to ensure producing thermal fully developed flow condition at the exist.
  The tube wall were heated electrically by the welder which provides high current and low voltages to keep the uniform heat flux condition in a inner pipe flow.
  Convective heat transfer is independent of the horizontal axis in fully developed flow and constant heat flux condition.
  The wall temperature $(T_{w})$ at the end of this section were measured by PT-100.
  \item Thermal equalized part\\
  The third part of the test section is the thermall equalized part, which includes a Static mixture.
  The static mixture forms turbulent and vortex.
  Then, the thermal gradients of fluids become averaged, and bull temperature $T_{out}$ is measured.
\end{enumerate}

\begin{figure}[ht]
	\vspace{0zh}
	\begin{center}
		\includegraphics[width=1\linewidth]{fig/testsection.pdf}
		\vspace{-3zh}
		\caption{test section}
		\label{testsection}
	\end{center}
	\vspace{0zh}
\end{figure}

\subsection{Length-to-diameter ratio}
%Ref. Experiments in Pipe Flows at Transitional and Very High Reynolds Numbers p25-
The length-to-diameter ratio is an important parameter to achieve the fully developed turbulent condition in the test section.
The entrance section has an inner diameter of $d_{i}=12mm$, and the length of $L=1.2m$ which length-to-diameter ratio is $L/d_{i}=100$.
Patel et.al.\cite{Zanoun2009} showed suitable the length-to-diameter ratio for fully developed turbulent flows.
According to their study, they found that the minimum developing length of $L/d_{i}=70d_{i}$.
Therefore, the length-to-diameter ratio of the entrance section in this experimental is long enough to ensure the fully developed turbulent flow state.

\section{Conductive heat transfer equation at the pipe wall}
\section{Evaluation procedure}

\clearpage
\section{Uncertainty Analysis of Measurement}
Uncertainty of measurement comprises many components such as bias or systematic error and random error in measuring dimension.
The measured value for dimension is the sum of true value, bias or systematic error and random error as shown in equation \ref{uncertainty_components}.
\begin{equation}
    x = x_{true}+x_{bias}+x_{random}
    \label{uncertainty_components}
\end{equation}
The bias or systematic error may be evaluated from statical distribution and can be characterized by experimental standard deviation.
The uncertainty of measurement was discussed by using ``Guide to the Expression of Uncertainty in Measurement'' \cite{jcgm2008}.

As shown in Chapter ?? in section ??, Reynolds number $Re$, skin friction coefficient $C_{f}$ and Nusselt number $Nu$ was calcurated from following equations (\ref{re_detail})(\ref{cf})(\ref{nu}).
\begin{equation}
    Re = \frac{UL}{\nu} = \frac{4\rho \dot{m}}{\mu \pi d_{i}}
    \label{re_detail}
\end{equation}
\begin{equation}
    C_{f} = \frac{\tau_{w}}{\rho_{b}\frac{U_{b}^{2}}{2}}, \ \tau_{w}=\frac{\Delta P}{\Delta z}\frac{d_{i}}{4}, \ U_{b}=\frac{4\dot{m}}{\rho_{b}\pi d_{i}^{2}}
    \label{cf_detail}
\end{equation}
\begin{equation}
    C_{f} = \frac{\Delta p}{\Delta z} \frac{d_{i}^{5}\pi^{2}\rho_{b}}{32\dot{m}^{2}}
    \label{cf}
\end{equation}
\begin{equation}
   Nu = \frac{\alpha d_{i}}{\lambda} = \frac{q_{w}d_{i}}{\lambda \left( T_{w}-T_{out} \right)}, \ q_{w}=\frac{\dot{m}C_{p} \left( T_{out}-T_{in} \right)}{d_{i}\pi L_{heated}}
   \label{nu_detail}
\end{equation}
\begin{equation}
   Nu = \frac{\dot{m}C_{p,b} \left( T_{out} - T_{in} \right) }{\lambda_{b} \left( T_{w} - T_{out} \right) \pi L_{heated}}
   \label{nu}
\end{equation}
Each parameters effect measurement uncertainty in each sensitivity in sensors.
The right choice of measurement sensors enable us to reduce measurement uncertainty.
However, there is a limitation in reality.
Cramer-Rao lower bound (CRLB) theory indicates the smallest uncertainty limit of estimating variance of deterministic process.
The maximum possible error in equations (\ref{re_detail})(\ref{cf})(\ref{nu}) can be estimated from measurement data.
The uncertainty of skin friction coefficient, $C_{f}$ is calcurated by equation (\ref{delta_cf}).
\begin{equation}
   \Delta C_{f} = \sqrt{\sum \left( \frac{\partial C_{f}}{\partial X_{i}}\Delta X_{i} \right)^{2}}
   \label{delta_cf}
\end{equation}
The uncertainty of skin friction coefficient, $Nu$ is calcurated by equation (\ref{delta_nu}).
\begin{equation}
   \Delta Nu = \sqrt{\sum \left( \frac{\partial Nu}{\partial X_{i}}\Delta X_{i} \right)^{2}}
   \label{delta_nu}
\end{equation}
Here, $\frac{\partial C_{f}}{\partial X_{i}}$ and $\frac{\partial Nu}{\partial X_{i}}$ represents uncertainty elements and assumed to be statistically independent.
In addition, $\Delta X_{i}$ represents absolute error of each sensors.
The resulting total measurement uncertaity in skin friction coefficient is shown in equation (\ref{delta_cf})
\begin{align}
    \Delta C_{f} & = \sqrt{\left(\frac{\partial C_{f}}{\partial \Delta p}\Delta \left( \Delta p \right) \right)^{2} + \left(\frac{\partial C_{f}}{\partial \rho_{out}}\Delta \rho_{out} \right)^{2} + \left(\frac{\partial C_{f}}{\partial \dot{m}}\Delta \dot{m} \right)^{2}}\\
    & = C_{f} \cdot \sqrt{\left( \frac{\Delta \left(\Delta p \right)}{\Delta p}\right)^{2} + \left(\frac{\Delta \rho_{out}}{\rho_{out}}\right)^{2} + \left( 2 \frac{\Delta \dot{m}}{\dot{m}}\right)^{2}}
    \label{delta_cf}
\end{align}

The resulting total measurement uncertaity in Nusselt number is shown in equation (\ref{delta_nu})

\begin{align}
    \Delta Nu & =
    \sqrt{
        \begin{aligned}
        \left(\frac{\partial Nu}{\partial \dot{m}}\Delta \dot{m}\right)^{2} + & \left(\frac{\partial Nu}{\partial c_{p,out}}\Delta c_{p,out}\right)^{2} + \left(\frac{\partial Nu}{\partial \lambda_{out}}\Delta \lambda_{out}\right)^{2} +\\
        & \left(\frac{\partial Nu}{\partial T_{in}}\Delta T_{in}\right)^{2}
        + \left(\frac{\partial Nu}{\partial T_{out}}\Delta T_{out}\right)^{2}
        + \left(\frac{\partial Nu}{\partial T_{w}}\Delta T_{w}\right)^{2}
        \end{aligned}
    }\\
    & = Nu \cdot \sqrt{
        \begin{aligned}
        \left(\frac{\Delta \dot{m}}{\dot{m}}\right)^{2}
        + & \left(\frac{\Delta c_{p,out}}{c_{p,out}}\right)^{2}
        + \left(\frac{\Delta \lambda_{out}}{\lambda_{out}}\right)^{2}+\\
        & \left(\frac{\Delta T}{T_{in} - T_{out}}\right)^{2}
        + \left(\frac{\Delta T}{T_{out} - T_{w}}\right)^{2}
        + \left(\frac{\Delta T\left( T_{in} - T_{w}\right)}{\left( T_{in} - T_{out}\right)\left( T_{out}-T_{w} \right)} \right)^{2}
        \end{aligned}
        }
\end{align}


Table (\ref{absolute_error}) shows absolute error for each sensors.
As shown in material properties, density, heat capasity and heat conductivity are temperature dependence.
Therefore, those absolute error influenced by the sensitivity of the temperature sensor, PT100.
\begin{table}[h]
\centering
\caption{Absolute error of contribution factor of the Reynolds number, skin friction coefficient and Nusselt number.}
\vspace{1zh}
\label{absolute_error}
\begin{tabular}{llll} \toprule%\boldmath
{Contribution} & {Symbol}          & {Absolute error}   & {Unit}\\ \midrule
Mass flow rate & $\Delta \dot{m}$  & $0.20E-3\dot{m}$   & $kg\cdot s^{-1}$\\
Pressure       & $\Delta(\Delta p)$& $0.35E-3\Delta p$  & $Pa$\\
Temperature    & $\Delta T$        & 0.040              & $K$\\
Density        & $\Delta \rho_{out}$& $0.26E-1$         & $kg\cdot m^{-3}$\\
Thermal conductivity & $\lambda_{out}$& $2.4E-5$        & $W\cdot m^{-1}\cdot K^{-1}$\\
Heat capasity  & $\Delta c_{p,out}$& 0.18               & $J\cdot kg^{-1}\cdot K^{-1}$\\
\bottomrule
\end{tabular}
\end{table}

\clearpage
\subsection{Case study}
In this section, the author showed evaluation procedure to calcurate measurement uncertainty for Reynolds number, skin friction coefficient and Nusselt number, respectively.
For this case study, Table (\ref{case_study}) shows one of the measurement data in this study.

\begin{table}[h]
\centering
\caption{}
\vspace{1zh}
\label{case_study}
\begin{tabular}{lllllllllll} \toprule%\boldmath
{Case} & {$Pr_{w}$} & {$Re$}  & {$C_{f}$} & {$Nu$} & {$\dot{m}$} & {$\Delta p$} & {$T_{w}$} & {$\rho_{out}$} & {$\lambda_{out}$} & $c_{p,out}$\\ \midrule
Case A & 10.0 & 4100 & 0.0103 & 38.7 & 0.044 & 253.8 & 71.2 & 1041 & 0.420 & 3.55\\
Case B &&&&&&&&&&\\
\bottomrule
\end{tabular}
\end{table}

Measurement uncertainty for Reynolds number was shown in equation ---


Measurement uncertainty for skin friction coefficient was shown in equation (\ref{delta_cf}).
By substituting absolute error shown in table (\ref{absolute_error}) into the equation (\ref{delta_cf}), which yeilds
\begin{equation}
   \Delta C_{f} = C_{f} \cdot \sqrt{\left( \frac{0.35E-3\Delta p}{\Delta p}\right)^{2} + \left(\frac{0.26E-1}{\rho_{out}}\right)^{2} + \left( 2\ \frac{0.20E-3\dot{m}}{\dot{m}}\right)^{2}}
   \label{case_study_cf1}
\end{equation}
Then, by substituting measurement data shown in table (\ref{case_study})into equation (\ref{delta_cf}), measurement uncertainty for skin friction ocefficient is calcurated.
\begin{align}
    \Delta C_{f} & = 0.0103 \cdot \sqrt{\left( \frac{0.35E-3\cdot 253.8}{253.8}\right)^{2} + \left(\frac{0.26E-1}{1041}\right)^{2} +   \left( 2\ \frac{0.20E-3\cdot 0.044}{0.044}\right)^{2}}\\
    &= 0.0103 \cdot \sqrt{ \left( 0.00035 \right)^{2} + \left( 0.0004 \right)^{2} + \left( 0.000025 \right)^{2} }\\
    &= 5.5E-6
   \label{case_study_cf2}
\end{align}

Measurement uncertainty for Nusselt number was shown in equation (\ref{delta_nu}).
By substituting absolute error shown in table (\ref{absolute_error}) into the equation (\ref{delta_nu}), which yeilds
\begin{align}
    \Delta Nu & = Nu\cdot
    \sqrt{
    \begin{aligned}
        & \left(\frac{0.20E-3\dot{m}}{\dot{m}}\right)^{2}
        + \left(\frac{0.18}{c_{p,out}}\right)^{2}
        + \left(\frac{2.4E-5}{\lambda_{out}}\right)^{2}+\\
        & \left(\frac{0.0040}{T_{in} - T_{out}}\right)^{2}
        + \left(\frac{0.0040}{T_{out} - T_{w}}\right)^{2}
        + \left(\frac{0.0040\cdot \left( T_{in} - T_{w}\right)}{\left( T_{in} - T_{out}\right)\left( T_{out}-T_{w} \right)} \right)^{2}
    \end{aligned}
    }\\
    & = Nu\cdot
    \sqrt{
    \begin{aligned}
        & \left(\frac{\Delta \dot{m}}{\dot{m}}\right)^{2}
        + \left(\frac{\Delta c_{p,out}}{c_{p,out}}\right)^{2}
        + \left(\frac{\Delta \lambda_{out}}{\lambda_{out}}\right)^{2}+\\
        & \left(\frac{\Delta T}{T_{in} - T_{out}}\right)^{2}
        + \left(\frac{\Delta T}{T_{out} - T_{w}}\right)^{2}
        + \left(\frac{\Delta T\left( T_{in} - T_{w}\right)}{\left( T_{in} - T_{out}\right)\left( T_{out}-T_{w} \right)} \right)^{2}
        \end{aligned}
        }
\end{align}








\chapter{Experiments}
\section{Validity of Experimental and evaluation procedure}
\section{Experimental result and variation}
\subsection{Validation of experimental result for $Pr_{w}=7$}
\subsection{Validation of experimental result for $Pr_{w}=10$}


\begin{figure}[ht]
	\vspace{0zh}
	\begin{center}
		\includegraphics[width=0.9\linewidth]{fig/pr10_recf.pdf}
		\vspace{-1zh}
		\caption{The comparison between skin frictin coefficient $C_{f}$ and bulk Reynolds number $Re_{b}$ for $Pr_{w} = 10$}
		\label{pr}
	\end{center}
	\vspace{0zh}
\end{figure}

\begin{figure}[ht]
	\vspace{0zh}
	\begin{center}
		\includegraphics[width=0.9\linewidth]{fig/pr10_renu.pdf}
		\vspace{-1zh}
		\caption{The comparison between heat transfer coefficient $Nu$ and bulk Reynolds number $Re_{b}$ for $Pr_{w} = 10$}
		\label{pr}
	\end{center}
	\vspace{0zh}
\end{figure}



\begin{table}[h]
\centering
\caption{Summary of the experimental parameters for $Pr_{w}=10$ and performance of the skin friction coefficient\ $C_{f}$, Nusselt number\ $Nu$}
\vspace{1zh}
\label{pr10}
\begin{tabular}{SrrSlSSSS} \toprule%\boldmath
{{$Pr_{w}$}} & {$Re_{b}$} & {$C_{f}$} & {$Nu$} & {$T_{w}\ [^\circ C]$} & {$T_{b}\ [^\circ C]$}  & {$\Delta T\ [^\circ C]$}  & {$q_{el}\ [kW/m^{2}]$} & {$q_{hc}\ [kW/m^{2}]$} \\ \midrule
10.0 & 3981  & 0.00999 & 41.3  & 71.4 & 67.8 & 3.6  & 6.8  & 5.3  \\
10.1 & 4047  & 0.00988 & 42.2  & 70.8 & 66.9 & 3.9  & 7.4  & 5.8  \\
9.9  & 4128  & 0.00983 & 42.9  & 71.6 & 67.2 & 4.4  & 8.2  & 6.6  \\
10.0 & 3981  & 0.00999 & 41.3  & 71.4 & 67.8 & 3.6  & 6.8  & 5.3  \\
10.1 & 4047  & 0.00988 & 42.2  & 70.8 & 66.9 & 3.9  & 7.4  & 5.8  \\
9.9  & 4128  & 0.00983 & 42.9  & 71.6 & 67.2 & 4.4  & 8.2  & 6.6  \\
10.0 & 3981  & 0.00999 & 41.3  & 71.4 & 67.8 & 3.6  & 6.8  & 5.3  \\
10.1 & 4047  & 0.00988 & 42.2  & 70.8 & 66.9 & 3.9  & 7.4  & 5.8  \\
\bottomrule
\end{tabular}
\end{table}

\subsection{Validation of experimental result for $Pr_{w}=13$}
\subsection{Validation of experimental result for $Pr_{w}=13$}

\section{Discussion}
\subsection{Reproducibility}
\subsection{Secoundary flow}
\subsection{Influence of heat flux}
\begin{figure}[ht]
	\vspace{0zh}
	\begin{center}
		\includegraphics[width=1\linewidth]{fig/pr10heatflux_recfdt.pdf}
		\vspace{-3zh}
		\caption{The comparison between skin frictin coefficient $C_{f}$ and bulk Reynolds number $Re_{b}$ for $Pr_{w} = 10$ vary with temperature defference $\Delta T$ between a pipe wall $T_{w}$ and the bulk $T_{b}$.}
		\label{}
	\end{center}
	\vspace{0zh}
\end{figure}

\begin{figure}[ht]
	\vspace{0zh}
	\begin{center}
		\includegraphics[width=1\linewidth]{fig/pr10heatflux_renudt.pdf}
		\vspace{-3zh}
		\caption{The comparison between heat transfer coefficient $Nu$ and bulk Reynolds number $Re_{b}$ for $Pr_{w} = 10$ vary with temperature defference $\Delta T$ between a pipe wall $T_{w}$ and the bulk $T_{b}$.}
		\label{}
	\end{center}
	\vspace{0zh}
\end{figure}


\begin{table}[h]
\centering
\caption{Summary of the experimental parameters for $Pr_{w}=10$ and performance of the skin friction coefficient\ $C_{f}$, Nusselt number\ $Nu$ and temperature defference\ $\Delta T=T_{w}-T_{b}$.}
\vspace{1zh}
\label{heatflux}
\begin{tabular}{SrrSlSSSS} \toprule%\boldmath
{{$Pr_{w}$}} & {$Re_{b}$} & {$C_{f}$} & {$Nu$} & {$T_{w}\ [^\circ C]$} & {$T_{b}\ [^\circ C]$}  & {$\Delta T\ [^\circ C]$}  & {$q_{el}\ [kW/m^{2}]$} & {$q_{hc}\ [kW/m^{2}]$} \\ \midrule
10.0 & 4176  & 0.00977 & 43.5 & 71.4 & 67.2 & 4.2  & 8.1  & 6.5  \\
9.9  & 4029  & 0.01020 & 44.1 & 71.6 & 63.7 & 7.9  & 14.8 & 12.3 \\
10.0 & 4098  & 0.00968 & 45.6 & 71.3 & 62.2 & 9.1  & 16.8 & 14.6 \\
10.0 & 3948  & 0.00977 & 45.1 & 71.5 & 57.9 & 13.5 & 25.2 & 21.5 \\ \midrule
10.0 & 6496  & 0.00897 & 68.3 & 71.5 & 67.8 & 3.7  & 11.1 & 8.9  \\
10.0 & 6569  & 0.00869 & 71.0 & 71.3 & 65.6 & 5.7  & 16.8 & 14.1 \\
9.9  & 6423  & 0.00879 & 71.2 & 71.5 & 62.8 & 8.8  & 25.7 & 22.0 \\
10.0 & 6337  & 0.00920 & 71.3 & 71.4 & 61.1 & 10.2 & 30.0 & 25.6 \\ \midrule
10.0 & 8540  & 0.00836 & 83.9 & 71.5 & 69.2 & 2.2  & 8.7  & 6.6  \\
10.0 & 8194  & 0.00843 & 84.9 & 71.4 & 66.9 & 4.5  & 16.3 & 13.4 \\
10.0 & 8315  & 0.00847 & 87.2 & 71.4 & 65.4 & 6.0  & 21.9 & 18.5 \\
10.0 & 8347  & 0.00848 & 90.1 & 71.4 & 62.1 & 9.3  & 34.1 & 29.6 \\ \midrule
9.9  & 9508  & 0.00809 & 91.5  & 71.5 & 69.5 & 2.1  & 8.8  & 6.6  \\
10.0 & 9320  & 0.00819 & 93.6  & 71.4 & 67.5 & 3.9  & 15.6 & 12.8 \\
10.0 & 9518  & 0.00815 & 97.0  & 71.3 & 66.1 & 5.2  & 21.2 & 17.8 \\
9.9  & 9380  & 0.00818 & 97.8  & 71.7 & 63.6 & 8.1  & 32.4 & 27.8 \\ \midrule
10.0 & 11628 & 0.00748 & 107.4 & 71.5 & 69.8 & 1.7  & 8.8  & 6.5  \\
9.8  & 11445 & 0.00767 & 108.7 & 72.1 & 69.4 & 2.8  & 13.3 & 10.7 \\
10.0 & 11213 & 0.00774 & 110.5 & 71.5 & 66.4 & 5.0  & 23.1 & 19.6 \\
9.9  & 11373 & 0.00784 & 113.4 & 71.7 & 64.5 & 7.2  & 33.4 & 28.7 \\ \midrule
10.0 & 12548 & 0.00738 & 114.3 & 71.4 & 69.7 & 1.7  & 8.9  & 6.7  \\
10.0 & 12458 & 0.00746 & 116.8 & 71.4 & 68.6 & 2.8  & 14.6 & 11.7 \\
9.9  & 12457 & 0.00761 & 118.8 & 71.6 & 67.1 & 4.5  & 22.6 & 19.0 \\
9.9  & 12427 & 0.00759 & 121.0 & 71.6 & 64.7 & 6.8  & 33.6 & 29.1 \\ \bottomrule
\end{tabular}
\end{table}


\begin{table}[h]
\centering
\caption{Direct Numerical Simuration\ (DNS) and Large Eggy Simuration\ (LES) for $Pr_{w} = 10$ vary with heat flux $q_{w}=20\ [kW/m^{2}]$ and $q_{w}=?\ [kW/m^{2}]$}
\vspace{1zh}
\label{simuration}
\begin{tabular}{lllrrrllllrl} \toprule
{Type} & {$Pr_{w}$} & {$Re_{\tau}$}& {$Re_{b}$} & {$C_{f}$} & {$Nu$} & {$T_{w}\ [^\circ C]$} & {$T_{b}\ [^\circ C]$}  & {$\Delta T\ [^\circ C]$}  & {$q_{w}\ [kW/m^{2}]$} \\ \midrule
DNS & 10.0 & 360 & 4165  & 0.00904 & 43.8  & 71.2 & 58.5 & 13.1 & 20.0  \\
DNS & 10.0 & 360 & 4165  & ?       & ?     & 71.2 & ?    & ?    & 40.0  \\ \midrule
DNS & 10.0 & 500 & 6587  & 0.00851 & 69.0  & 71.2 & 63.3 & 8.3  & 20.0 \\
DNS & 10.0 & 500 & 6587  & ?       & ?     & 71.2 & ?    & ?    & 40.0 \\ \midrule
LES & 10.0 & 600 & 8498  & 0.00779 & 84.8  & 71.2 & 64.8 & 6.8  & 20.0 \\ \midrule
LES & 10.0 & 750 & 11465 & 0.00704 & 106.1 & 71.2 & 66.2 & 5.4  & 20.0 \\ \bottomrule
\end{tabular}
\end{table}

\subsection{Scattering and proberbility density function}
\subsection{Comparison with Bertsche}
\subsection{Comparison with DNS and LES}
\subsection{Effect of the presure popower}
\subsection{Effect of material properties}
\subsection{}
\chapter{Conclusion}


\appendix
% !TEX root = ../thesis-sample.tex
\appendix
\doublespacing
\chapter{Material properties}
A 50/50vol\% mixture of water and glycole which is a typical liquid coolant in automotive applications were used as a operating fluid.

\chapter{Post processing}
\lipsum[24]


%http://lightology.hatenablog.com/entry/2018/02/12/221721
\bibliographystyle{amsplain}
\bibliography{/Users/Shared/TeXLive/texmf/bibtex/bib/local/graz.bib}
%\bibitem{Frank}Frank P.Incropera et al., ``Fundamentals of Heat and Mass Transfer'', (Wiley, 2006)

\end{document}
