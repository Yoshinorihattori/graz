
\section{研究背景}
近年のコンピューター,スマートフォン等の電子デバイスにおいて半導体素子の小型化・配線技術の向上が飛躍的に進んでいる.
.

\section{先行研究}
中別府ら\cite{Nakabeppu2014}により単一沸騰気泡の除熱効果について研究が行われているが,気泡の大きさを制御されていない.
井上ら\cite{Inoue2016}により単一沸騰気泡が生成できる機構がある.
具体的にはMEMS技術を用いて伝熱面上の過熱度を制御できるヒータと単一気泡が生成できる発泡機構を有する伝熱面(以下,MEMS伝熱面)が作成された.
この研究では単一沸騰気泡発泡までの過熱時間を制御することにより気泡成長の制御を行なった.

\section{研究目的}
本稿では「単一沸騰気泡発泡までの過熱時間」と「伝熱面上の過熱度」の両方の制御を行い,その時の単一沸騰気泡成長を制御する.
2つのパラメータを制御することにより,過熱液層が単一沸騰気泡に与える影響をより詳細に検討することを目的とする.
気泡成長の制御から感温塗料(温度により輝度が変化する塗料)を用いて温度分布を測定することにより除熱効果の高い単一沸騰気泡解明の将来の道筋として役立てることを目的とする.

\section{構成}
本論文は本章を含めて全5章から構成されている.
第1章では研究背景と先行研究,研究目的について述べた.
第2章ではMEMS伝熱面と実験装置と実験方法について述べる.
第3章では実験結果について述べる.
第4章では考察と結論について述べる.
%第5章では今後の展望について述べる.
また,付録としてMEMS伝熱面の作成手順とスパッタ装置と制御装置プログラミングについて述べる.
