
\section{単一沸騰気泡生成の原理}
一般的な沸騰現象と本研究の単一沸騰気泡生成は原理が異なる.
Fig. \ref{NormalBoillingProcess}に一般的な沸騰気泡の気泡成長についてモデル図を示す.
一般的な沸騰気泡は,伝熱面上の微細な傷(キャビティ)にトラップされている気泡を核として気泡が生成されていると考えられている.
この気泡成長はキャビティ形状,伝熱面上の過熱度の2つに依存していると考えられており,制御することが難しい.
また,電気ケトルなどの身近な沸騰現象に見てわかるように沸騰気泡がランダムかつ膨大に発生し,観察することが難しい.
\begin{figure}[ht]
\vspace{0zh}
 \begin{center}
  \includegraphics[width=0.7\linewidth]{pdf/thesis2017_NormalBoillingProcess}
  \vspace{0zh}
  \caption{一般的な沸騰気泡の気泡成長}\label{NormalBoillingProcess}
 \end{center}
 \vspace{-1zh}
\end{figure}

そこで本研究では沸騰現象を単純化するために一つの沸騰気泡に着目する.
Fig. \ref{HowToCreateSingleBubble}に本研究での単一気泡を生成する仕組みについて示す.
Step1で平滑なシリコンウェハ上にCr(クロム)で電極(以下,トリガー)をMEMS技術を用いて作製する.(詳しい作成手順については次項で述べる.)
Step2でトリガーの反対側から薄膜ヒーターを用いて過熱し,トリガー側に過熱液層を形成する.
Step3で薄膜ヒーター過熱時間(待ち時間)より過熱液層(液層が加熱され,飽和温度以上に熱せられ薄い層)の大きさを制御する.
また気泡成長画像を取得するためハイスピードカメラの撮影を開始する.
Step4で水の電気分解を用いて微細な水素気泡を発生させる.
この微細な水素気泡が単一沸騰気泡の核となる.
薄膜ヒータで加熱してから水素気泡を発生させるまでの時間(待ち時間)と薄膜ヒーターがトリガー面に伝える熱流束の2つのパラメータを制御することにより単一沸騰気泡の成長を制御する.
%制御方法の詳しい説明は次項でのべる.

また,Crを用いた水の電気分解の反応式は
\begin{align}
 % 複数行に渡る数式
  \text{陽極:} && \frac{2}{3} {\rm Cr} &\rightarrow \frac{2}{3}{\rm Cr^{3+}} + 2e^-\\
  \text{陰極:} && {\rm 2H_2O} + 2{\rm e}^- &\rightarrow {\rm H_2} + {\rm 2OH^{-}}
\end{align}
である.
この電気分解により陽極から水素気泡が発生し,単一沸騰気泡の核となる.

\begin{figure}[ht]
\vspace{1zh}
 \begin{center}
  \includegraphics[width=1\linewidth]{pdf/thesis2017_HowToCreateSingleBubble}
  \vspace{-1.5zh}
  \caption{本研究での単一沸騰気泡の生成原理}\label{HowToCreateSingleBubble}
 \end{center}
 \vspace{0zh}
\end{figure}
%
%\section{MEMS伝熱面の作製方法}
%
%伝熱面はスパッタリングと呼ばれる薄膜形成技術と紫外線硬化樹脂(以後これをレジストと呼ぶ)を用いたフォトリソグラフィによって作製した.
%スパッタリングとは真空中で行う薄膜形成技術である.
%薄膜として堆積させたい金属をターゲットとして設置し,高電圧をかけてイオン化させた希ガス元素(Ar)を衝突させ,弾き飛ばされた金属原子が基盤に到達して製膜する技術である.
%
%\begin{figure}[ht]
%\vspace{-1zh}
% \begin{center}
%  \includegraphics[width=0.7\linewidth]{chuukanzu6.pdf}
%  \vspace{-1zh}
%  \caption{スパッタリングのイメージ図}\label{apparatus}
% \end{center}
% \vspace{-2zh}
%\end{figure}
%
%また,今回使用するレジストはポジ型と呼ばれるものであり,紫外線を露光した部分が現像することによって溶けてなくなるという性質がある.
%今回使用するフォトリソグラフィ技術は2つある.
%1つはフォトレジストで保護されていない剥き出しの金属部分を薬剤で溶かして加工するウエットエッチングである.
%もう1つはフォトレジストパターンの上から,スパッタリングで薄膜を形成し,フォトレジストごと除去するリフトオフ法である.
%この二つの加工技術を使用し,単一沸騰気泡生成装置の作成を行う.
%\begin{figure}[ht]
%\begin{center}
%  \includegraphics[width=0.8\linewidth]{chuukanzu11.pdf}
%  \vspace{-1zh}
%  \caption{今回使用するフォトリソグラフィ技術}\label{4}
% \end{center}
% \vspace{-1.5zh}
%\end{figure}
%\vspace{-0.5zh}
%\clearpage
%

\section{MEMS技術を用いた単一沸騰気泡発泡装置と薄膜ヒーター}
%Fig.  \ref{4}に
MEMS技術を用いた単一沸騰気泡発泡装置と過熱液層を形成するための薄膜ヒーター(以下,MEMS伝熱面)を井上ら\cite{Inoue2016}の研究を参考にして作成する.
平滑面である両鏡面加工されたシリコンウェハ(単結晶シリコン,三菱マテリアルトレーディング株式会社)にスパッタリングと呼ばれる薄膜形成技術と紫外線硬化樹脂(以下,レジスト)を用いたフォトリソグラフィによってMEMS伝熱面を作成する.

単一沸騰気泡生成装置のトリガーとなるCr電極先端部,配線接続部以外は1層目,3層目のSiO$_{2}$(二酸化ケイ素)で覆われ,試験液である純水や伝熱面との絶縁を図る.
配線接続部は最終的に樹脂で覆うことにより水との絶縁をする.
Crは導電性であり,SiO$_{2}$(二酸化ケイ素)は絶縁を図るために用いる.

スパッタリング,詳しいMEMS伝熱面の作成手順ついては付録に示す.
%Fig. \ref{6}に実際に作成した薄膜ヒーター,Fig. \ref{7} に単一沸騰気泡発泡装置を示す.
単一沸騰気泡発泡装置は2本の電極から水の電気分解を用いて水素気泡を発生させる.
%寸法などの細かい説明
薄膜ヒータはMEMS伝熱面上に作成する蛇行回路である.
%寸法などの細かい説明
蛇行回路の抵抗値は概ね1.5kohmである.
%\clearpage

%\begin{figure}[ht]
%\begin{center}
%  \includegraphics[width=0.8\linewidth]{pdf/thesis2017_Mems.pdf}
%  \vspace{-1zh}
%  \caption{MEMS伝熱面の概略図}\label{4}
% \end{center}
% \vspace{-1.5zh}
%\end{figure}
%\vspace{-0.5zh}
%%\clearpage
%
%%\begin{figure}[ht]
%%\vspace{-0.5zh}
%%\begin{center}
%%  \includegraphics[width=1\linewidth]{abc.pdf}
%%  \vspace{-1zh}
%%  \caption{MEMS伝熱面の作成手順 }\label{5}
%% \end{center}
%% \vspace{-0.5zh}
%%\end{figure}
%
%\begin{figure}[ht]
%\vspace{-0.5zh}
%\begin{center}
%  \includegraphics[width=0.8\linewidth]{pdf/thesis2017_Heater.pdf}
%  \vspace{-0.5zh}
%  \caption{ヒータ面}\label{6}
% \end{center}
% \vspace{-0.5zh}
%\end{figure}
%
%\begin{figure}[ht]
%\vspace{-0.5zh}
%\begin{center}
%  \includegraphics[width=0.8\linewidth]{pdf/thesis2017_Trriger.pdf}
%  \vspace{-2zh}
%  \caption{発泡機構面}\label{7}
% \end{center}
% \vspace{-0.5zh}
%\end{figure}
%\clearpage


\section{テストセクション}
図\ref{TestSection}にテストセクションの概略図を示す.
MEMS伝熱面上のトリガ,薄膜ヒータの配線がされている.
シリコンウェハは撥水性が高く,ハンダ付によって直接配線をすることができない.
そこで抵抗が低く,接着剤の役割をする銀ペースト(ドータイト導電性ペースト,藤倉化成株式会社)を用いてハンダの相性の良い銅板を介して配線を行う.
蛇行回路の抵抗値は概ね1.5kohmであるが,銀ペーストの抵抗が大きいとヒーター加熱の際,銀ペースト部が発熱してしまう可能性があるので注意が必要である.
銀ペーストの抵抗値を出来るだけ小さくする方法として接着後,ヒーター加熱時に銅板を押し付けると抵抗値が低くなる傾向がある.
それでも銀ペーストの抵抗が大きかった場合,電極部の空いている面に再度,銅板を接着する必要がある.

配線されたMEMS伝熱面は耐熱性樹脂PEEK(有限会社栗本加工)に耐水性樹脂(1液型 RTVゴム脱アセトン型,信越化学工業株式会社)を用いて固定する.
樹脂硬化には自然乾燥で1〜2日要する.
ハイスピードカメラにて気泡画像を取得する際,耐熱性樹脂を盛りすぎると光源の影により気泡成長測定が困難となる可能性があるので注意が必要である.
\begin{figure}[ht]
\vspace{0zh}
\begin{center}
  \includegraphics[width=0.5\linewidth, angle=270]{pdf/thesis2017_TestSection.pdf}
  \vspace{0zh}
  \caption{テストセクション}\label{TestSection}
 \end{center}
 \vspace{-0.5zh}
\end{figure}
\clearpage


\section{実験装置}
図\ref{Chamber}にチャンバーの概略図を示す.
チャンバー内を純水で満たし,テストセクションを設置する.
それに加え純水を飽和状態に加熱するためにコイル状に加工した補助ヒーター(M-2型フレキヒータ,坂口電熱株式会社)を設置する.
試験液体が飽和状態にならないとサブクール状態になり気泡成長がスムーズに行われない.
そのため図\ref{Insulation}に示すようにチャンバー内をコルクを用いて断熱を行う.
チャンバー上部は穴が空いており大気開放となっている.


\begin{figure}[ht]
\vspace{0zh}
\begin{center}
  \includegraphics[width=1.0\linewidth]{pdf/thesis2017_Chamber.pdf}
  \vspace{-2zh}
  \caption{実験装置概略図}\label{Chamber}
 \end{center}
 \vspace{-0.5zh}
\end{figure}
\clearpage

\begin{figure}[ht]
\vspace{-0.5zh}
\begin{center}
  \includegraphics[width=0.5\linewidth, angle=270]{pdf/thesis2017_Insulation.pdf}
  \vspace{0zh}
  \caption{断熱材}\label{Insulation}
 \end{center}
 \vspace{-0.5zh}
\end{figure}
%\clearpage

図\ref{ExperimentalSetup}に実験装置図を示す.
テストセクションに配線されているMEMS伝熱面は制御装置(Arduino)によってヒーター加熱時間(待ち時間)と単一沸騰気泡生成装置の制御が行われる.
印可する電圧はそれぞれ直流安定化電源(EX1125H2 および EX750U2,株式会社高砂製作所)で制御を行う.
気泡画像はバックライト法によりハイスピードカメラ(IDF-Express,株式会社フォトロン)で撮影する.
ハイスピードカメラは焦点距離50$ mm $の標準レンズ(Nikkor 50$ mm $ f/1.2,株式会社ニコン)と,焦点距離20$ mm $の接写リング(株式会社ニコン)を取り付ける.
光源はファイバーライト(DC950 ILLUMINATOR,Dolan Jenner)を用いる.
またデータロガーを用いて制御の確認を行う.

\begin{figure}[ht]
\vspace{2zh}
\begin{center}
  \includegraphics[width=0.8\linewidth]{pdf/thesis2017_ExperimentalSetup.pdf}
  \vspace{0zh}
  \caption{実験装置概略図}\label{ExperimentalSetup}
 \end{center}
 \vspace{1zh}
\end{figure}
\clearpage



\section{実験方法}
チャンバー内の温度は補助ヒータで飽和温度に保つ.
補助ヒータは電圧調整器(山菱電機株式会社)によって80$ V $に設定をし,温度調節器(NC-010D,株式会社日伸理化)を105$ ℃ $に設定し,脱気を行う.
脱気後もMEMS伝熱面上に気泡が存在する場合は,さらに脱気を続けるか,攪拌器により除去する.
ハイスピードカメラの焦点はMEMS伝熱面の発泡機構の電極先端に合わせる.
最初に印をつけた十字のアライメントマークを見つけると焦点を合わせやすい.

薄膜ヒータ加熱後,待ち時間$\Delta t$ 後に3$ ms $間電極に印加する.
ハイスピードカメラは電極に印加する10$ ms $前に撮影を開始する.
気泡画像を取得するハイスピードカメラの撮影速度は1000〜4000$ fps $の間で行なった.
気泡径 $d$ は取得した気泡画像の最大幅を採用した.


%$\Delta t$ は50$ ms $, 150$ ms $, 200$ ms $, 250$ ms $,300$ ms $, 350$ ms $, 400$ ms $, 450$ ms $, 500$ ms $, 550$ ms $, 600$ ms $, 650$ ms $, 700$ ms $ の計13回計測した.
%熱流束は$q=49.4\,\rm kW/m^2$に設定した.













