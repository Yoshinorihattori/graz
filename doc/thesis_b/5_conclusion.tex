

\section{結論}
本研究では異なる3つの熱流束における沸騰気泡成長制御を行なった.
だが,実験で確認された単一沸騰気泡が水素気泡を核として発生したものなのか,突沸により発生したのか確認ができなかった.原因はヒーター加熱の時間すなわち待ち時間が不確かであることである.
その待ち時間を正確に判断するためにノイズ対策,時間の正確性,突沸の3つの原因を探り,正確な待ち時間を得るための術を用意した.


%本研究では,沸騰気泡の成長を制御することを目的として,気泡核が供給できる機構および伝熱面を加熱できる薄膜ヒータをシリコンウェハ上に作製し,ヒータ加熱から気泡核を供給するまでの時間 $\Delta t$ を制御することで沸騰気泡の成長を制御した.
%現象の時間スケールに対して十分な精度で発泡のタイミングで制御できており,$\Delta t$ を変えることで沸騰気泡の成長速度や離脱気泡径を変えることができた.


\section{今後の課題}
任意の熱流束・待ち時間において気泡成長画像を取得する.
これにより詳細な熱流束・待ち時間の気泡成長速度(横軸時間,縦軸気泡径)への依存性がわかる.
本実験ではMEMS電熱面に流す電力全てが過熱液層に伝えられると仮定して実験を行なっている.
MEMS電熱面から液体に伝わる熱損失を考慮するとより正確な熱流束を求めることができる.
Moghaddamら\cite{Moghaddam2006}によると薄膜ヒーター面から液体の熱損失はシリコンウェハを通しての熱損失と比較し数十%だと示されている.
また,TSP(感温塗料)を用いて気泡接触面の温度場を計測する.
これにより再濡れ面・乾き面における温度上昇箇所がわかり,除熱効率の良い単一沸騰気泡の発生条件について詳細なデータが得られる.
%今後は気泡の大きさごとの伝熱面温度の変動を調べ,最も除熱効率の良い沸騰気泡発生の条件について研究していく.



