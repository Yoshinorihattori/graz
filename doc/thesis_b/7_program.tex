\begin{figure}[ht]
\vspace{0zh}
\begin{center}
\includegraphics[width=1\linewidth]{pdf/thesis2017_program.pdf}
\vspace{0zh}
%\caption{}\label{Arduino_heater}
\end{center}
\vspace{0zh}
\end{figure}

%\begin{figure}[ht]
%\vspace{0zh}
%\begin{center}
%\includegraphics[width=0.65\linewidth]{figures/cnhb2.pdf}
%\vspace{0zh}
%%\caption{}\label{Arduino_heater}
%\end{center}
%\vspace{0zh}
%\end{figure}

%\newpage
%
%このプログラムは待ち時間 $t_{w}$ = 50 ms,発泡トリガへの印加時間 3 ms である.
%高速度カメラのトリガはカメラへの電圧が0になった時である.
%$t_{w}$ = 50 ms の場合,薄膜ヒータに印加してから5 ms 後に高速度カメラへ電圧がかかり,40 ms 後にカメラへの電圧を0にし,その5 ms 後に発泡トリガに印加する.
%したがって,発泡トリガに印加されるまでが 5 + 40 + 5 = 50 ms となる.
%また,発泡トリガに印加されてからも薄膜ヒータに電圧がかかっている.
