\textbf{<配線方法>}

MEMS伝熱面の配線は導線とはんだで繋ぐことで可能とする.
しかし,シリコンウェハは撥水性が高く,はんだとの相性が悪い.
そこで,はんだと相性の良い銅板と,抵抗が低く接着剤の役割をする銀ペーストを使用し,はんだで導線をつなぎ合わせられる土台を作る.

\begin{enumerate}

\item 銅板をハサミを用いて約5 mm × 5 mm (導線をはんだできる)程度切り出す.
 
\begin{figure}[ht]
\vspace{0zh}
\begin{center}
\includegraphics[width=0.7\linewidth]{douban.pdf}
\vspace{0zh}
\caption{銅板の切り出し}\label{ wafer}
\end{center}
\vspace{0zh}
\end{figure} 
\item ハンマーを用いて切り出した銅板を平らにする.さらにそれをIPA(イソプロピルアルコール)を染み込ませたベムコットンで汚れをふき取る.
 
\begin{figure}[ht]
\vspace{0zh}
\begin{center}
\includegraphics[width=0.6\linewidth]{hanma.pdf}
\vspace{0zh}
\caption{銅板を平らにする}\label{ wafer}
\end{center}
\vspace{0zh}
\end{figure}



 
\item 銀ペースト(ドータイト導電ペースト,藤倉化成株式会社)を常温に戻す.2液タイプのため、同量を混ぜ合わせる.
 

\begin{figure}[ht]
\vspace{0zh}
\begin{center}
\includegraphics[width=1\linewidth]{Ag.pdf}
\vspace{-1zh}
\caption{銀ペースト}\label{ wafer}
\end{center}
\vspace{0zh}
\end{figure}

\item MEMS伝熱面の配線部分4箇所に,爪楊枝の柄の部分で銀ペーストを擦り込むように塗布する.
\item 塗布した銀ペーストが乾く前に銅板を擦り付けるように乗せ,できるだけ密着させる.

\begin{figure}[ht]
\vspace{0zh}
\begin{center}
\includegraphics[width=1\linewidth]{Agtohu.pdf}
\vspace{-1zh}
\caption{MEMS伝熱面の配線部に銅板をつける}\label{wafer}
\end{center}
\vspace{0zh}
\end{figure}

\item 155 ℃に設定したホットプレートでMEMS伝熱面を30分間加熱する.熱が均一になるように加熱中は蓋をする.

\begin{figure}[ht]
\vspace{0zh}
\begin{center}
\includegraphics[width=0.7\linewidth]{Agheat.pdf}
\vspace{0zh}
\caption{銀ペーストを硬化}\label{ wafer}
\end{center}
\vspace{0zh}
\end{figure}


\end{enumerate}

\clearpage

\textbf{<MEMS伝熱面の設置>}

MEMS伝熱面を水槽に設置する方法について述べる.
ポリカーボネートにMEMS伝熱面を設置し,それを水槽の底に設置する.
その際には水漏れがないよう,樹脂やゴムシートを用いて組み立てる.
この工程で導線のはんだも行う.

\begin{enumerate}

\item Fig. \ref{12}のように加工したポリカーボネートを使用する.中央の正方形の穴にMEMS伝熱面を設置する.外側の4箇所の穴は水槽の底に固定するネジ穴,それ以外の2箇所の穴はMEMS伝熱面からのびる導線を通す穴である.
 
 
\begin{figure}[ht]
\vspace{0zh}
\begin{center}
\includegraphics[width=0.7\linewidth]{pori.pdf}
\vspace{-3zh}
\caption{MEMS伝熱面を設置するポリカーボネート}\label{12}
\end{center}
\vspace{0zh}
\end{figure}


\item ポリカーボネートを水槽の底にボルトで設置し,点対称順にボルトを締めていく.
\item MEMS伝熱面をポリカーボネートに乗せ,カプトンテープで固定する.
\item 水槽の底の端を何かに乗せ,2箇所の穴から電極側の導線を通してから,配線部分のはんだをする.


 \begin{figure}[ht]
\vspace{0zh}
\begin{center}
\includegraphics[width=0.7\linewidth]{poriset.pdf}
\vspace{-5zh}
\caption{ MEMS伝熱面とポリカーボネートを水槽の底に設置}\label{ wafer}
\end{center}
\vspace{0zh}
\end{figure}


\item 樹脂(Silicone COMPONENT RTV, Shin-etsu)を爪楊枝を用いて,ポリカーボネートの隙間および配線まわりに塗りこむことで,水漏れや配線部の漏電防止を行う.
\item 樹脂を固めるため,埃がつかないよう箱を被せ,通気良い状態で1日放置する.

 
  \begin{figure}[ht]
\vspace{0zh}
\begin{center}
\includegraphics[width=1\linewidth]{jyusi.pdf}
\vspace{-1zh}
\caption{樹脂塗り}\label{ wafer}
\end{center}
\vspace{0zh}
\end{figure}


\item 水槽の隙間にゴムシートを挟みながら組み立てれば,実験系の設置が完了である.ネジは点対称順に止めること.

 
  \begin{figure}[ht]
\vspace{0zh}
\begin{center}
\includegraphics[width= 0.5\linewidth]{setOK.pdf}
\vspace{0zh}
\caption{完成した実験系}\label{ wafer}
\end{center}
\vspace{0zh}
\end{figure}


\end{enumerate} 
