%\section{MEMS伝熱面の作成手順}
 \textbf{<ウェハの切り出し>}

\begin{enumerate}

\item 両面鏡面加工されたシリコンウェハ(単結晶シリコン,三菱マテリアルトレーディング株式会社)を35 mm × 35 mm の正方形を4枚切り出す.ウエハを傷つけないためにIPA(イソプロピルアルコール)を染み込ませたベムコットンでウェハ・定規・クラフトボードをこまめにふき取る.
このシリコンウェハは両面に厚さ3000\AA\textpm10\% の酸化膜が形成されている.


\item ガラスシャーレにベムコットンを敷き,切り出したウェハを重ならないように並べ,IPAで浸し超音波洗浄を用いて5分間洗浄を行う.
 
\item 洗浄が終わったウェハをアセトンを染み込ませたベムコットンで1枚ずつ丁寧に拭く.水垢が残っていると金属粒子の密着が悪くなる.
 
\item ウェハの片面にエッチング液などから保護するためのウェハ保護テープ(SP 594M 130,古河電気工業株式会社)を貼り付ける.この保護テープは紫外線で露光すると硬化し,粘着力を失って剥がれるので取り扱いには注意が必要である.

\clearpage
\end{enumerate}

\newpage

\textbf{<a:薄膜ヒータ面のスパッタリング>}

\begin{enumerate}

\item スパッタ装置(CFS-4ES-231,芝浦メカトロニクス株式会社)を説明書通り立ち上げる.芝浦工業大学においてスパッタ装置の使用は認定ユーザの立会いのもと使用する(2017年12月 現在).また酸素濃度が薄くなる可能性があるため2人以上で作業する.
 
\item スパッタ装置,ウェハの設置はカプトンテープを用いる.ウェハの端をカプトンテープで貼り固定する.

\item スパッタ装置の出力を400 W に設定し,5分間SiO$_{2}$をスパッタする.
\item スパッタ装置の出力を400 W に設定し,5分間Crをスパッタする.
\item スパッタ装置を説明書通り立ち下げる.
 \clearpage
\end{enumerate}
 
 \newpage
\textbf{<b:薄膜ヒータのフォトパターン作製>}

\begin{enumerate}

\item スピンコータの中心にウェハをカプトンテープで固定する.
\item ポジ型レジスト(ma-P 1275,株式会社オーエステック)をマイクロピペット(Nichipet EX,株式会社ニチリョー)を用いて700 \textmu l 滴下する.
%このポジ型レジストは粘性が420 mPa・s とオリーブ油と卵黄の間程度であり,保管温度は18~25℃となっている.そのため冷蔵庫などに入れずに常温で保存する.このレジストの露光に必要な光量は430\textpm20 mJ/cm$^{2}$ である.

\item 滴下したレジストを爪楊枝を用いてウェハ全体に伸ばす.このとき気泡が残るとムラができるので注意が必要である..
 
\item スピンコータを用いて2500 rpm で30秒間回転させる.回転数は1秒に100 rpm 程度を目安にゆっくりと上げる.回転数を下げるときも同様である.スピンコータは必ず毎回ベムコットンに染み込ませたアセトンを用いて洗浄する.
 
\item 100 ℃に設定したホットプレートでウェハを10分間加熱する.熱が均一になるように加熱中は蓋をする.
\item フォトマスク(株式会社セブンクリエイト)を露光装置にカプトンテープを用いて設置する.ヒータのパターンは線幅200 \textmu m,間隔200 \textmu m の蛇行回路である.ヒータの抵抗値はスパッタの時間で変わってくるが,本手順では概ね1.5 kΩ 程度となる.

\item 露光装置を出力100 \% で30秒間露光し,出力0 \% で10秒ほど放置する.これを1セットとして4セット行う.その後出力100 \% で20秒間露光し,出力0 \% で10秒ほど放置する.これを2セット行う.ただし,露光時間は光源までの距離や露光装置の経年劣化も影響するため,露光の出力を測るパワーチェッカを用いて必要な光量になるよう適宜調節する.露光が足りないとパターンよりひとまわり大きいパターンが形成される.逆に露光過多だとパターンのエッジ部分がギザギザに形成されてしまう.

\item 現像液(ma-D 311,株式会社オーエステック)をウェハが浸るほどの量をガラスシャーレに入れる.ガラスシャーレを約80秒間,手で揺らし現像する.必要に応じて時間は調節する.

\item 現像液からウェハを取り出し,純水でよくすすぐ.その後エアダスタで水気を飛ばす.廃液は各廃液タンクに漏斗を用いて入れる.この段階でレジストがヒータのパターン(蛇行回路)になっている.
\end{enumerate} 

\clearpage




\textbf{<c:薄膜ヒータのCrエッチング>}

\begin{enumerate}
\item Crエッチング液(クロム膜エッチング液N14B,日本化学産業品株式会社)をガラスシャーレに深さ10 mm 程度になるように入れる.ウェハを浸し約90秒間,手で揺らしエッチングする.

\item エッチング液からウェハを取り出し,純水でよくすすぐ.その後エアダスタで水気を飛ばす.廃液は各廃液タンクに漏斗を用いて入れる.
\item アセトンを染み込ませたベムコットンでCrを保護していたレジストを拭いて除去する.この段階でヒータは完成である.


\end{enumerate} 
\vspace{5zh}

\textbf{<d:電極面のスパッタリング>}

\begin{enumerate}

\item 片面に貼ってある保護テープに紫外線を出力100 \% で5分間照射し,テープを剥がす.
\item 新たに保護テープをヒータ面に貼る.これによりヒータを保護する.
\item スパッタ装置を説明書通り立ち上げる. 
\item ウェハの設置はカプトンテープを用いる.ウェハの端をカプトンテープで貼り固定する.
\item スパッタ装置の出力を400 W に設定し5分間SiO$_{2}$をスパッタする.
\item スパッタ装置の出力を400 W に設定し7分間Crをスパッタする.
\item スパッタ装置を説明書通り立ち下げる.


\clearpage
\end{enumerate} 

\newpage
\textbf{<e:電極のフォトパターン作製>}

\begin{enumerate}

\item スピンコータの中心にウェハをカプトンテープで固定する.
\item ポジ型レジストをマイクロピペットを用いて700 \textmu l 滴下する.
\item 滴下したレジストを爪楊枝を用いてウェハ全体に伸ばす.
 
\item スピンコータを用いて2500 rpm で30秒間回転させる.回転数は1秒に100 rpm 程度を目安にゆっくりと上げる.回転数を下げるときも同様である.
\item 100 ℃に設定したホットプレートでウェハを10分間加熱する.熱が均一になるように加熱中は蓋をする.
\item フォトマスクを露光装置にカプトンテープを用いて設置する.発泡トリガのパターンは対向する線幅150 \textmu m の電極と配線端子である.電極間の距離は20 \textmu m である.

\item 露光装置を出力100 \% で30秒間露光し,出力0 \% で10秒ほど放置する.これを1セットとして4セット行う.その後出力100 \% で20秒間露光し,出力0 \% で10秒ほど放置する.これを2セット行う.

\item 現像液をウェハが浸るほどの量をガラスシャーレに入れる.ガラスシャーレを約80秒間,手で揺らし現像する.
\item 現像液からウェハを取り出し,純水でよくすすぐ.その後エアダスタで水気をなくす.


\end{enumerate} 

\vspace{5zh}
\textbf{<f:電極面のCrエッチング>}

\begin{enumerate}
\item Crエッチング液をガラスシャーレに深さ10 mm 程度になるように入れる.ウェハを浸し約90秒間,手で揺らしエッチングする.
\item エッチング液からウェハを取り出し,純水でよくすすぐ.その後エアダスタで水気をなくす.この段階でウェハ上にはCrで電極が作成されている.


\end{enumerate} 

\vspace{5zh}
\textbf{<g:絶縁層のフォトパターン作製>}

\begin{enumerate}
\item 電極が作成されたウェハをスピンコータの中心にカプトンテープで固定する.
\item ポジ型レジストをマイクロピペットを用いて700 \textmu l 滴下する.
\item 滴下したレジストを爪楊枝を用いてウェハ全体に伸ばす.
 
\item スピンコータを用いて2500 rpm で30秒間回転させる.回転数は1秒に100 rpm 程度を目安にゆっくりと上げる.回転数を下げるときも同様である.
\item 100 ℃に設定したホットプレートでウェハを10分間加熱する.熱が均一になるように加熱中は蓋をする.
\item ガラスマスク(ミタニマイクロニクス株式会社)をカプトンテープで露光装置に設置する.ガラスマスクが汚れている場合はIPAを染み込ませたベムコットンで拭く.
\item 露光装置に取り付けてあるUSB顕微鏡(秀マイクロン4,株式会社テック)を Mac Book に接続し,アプリPhoto Boothで起動する.このUSB顕微鏡の先端にはLEDライトが取り付けてあるためこのまま使用するとレジストが露光してしまう可能性がある.そのためUSB顕微鏡先端にイエローフィルムが取り付けてある.Crで作成された電極の先端部分にガラスマスクの黒い丸(直径 100 \textmu m)が重なるように微動装置を用いて微調整する.


%\begin{figure}[ht]
%\vspace{0zh}
%\begin{center}
%\includegraphics[width=1\linewidth]{glassmask.pdf}
%\vspace{0zh}
%\caption{リフトオフ法に使用するガラスマスク}\label{glassmask}
%\end{center}
%\vspace{0zh}
%\end{figure}


\item 露光装置を出力100 \% で30秒間露光し,出力0 \% で10秒ほど放置する.これを1セットとして4セット行う.その後出力100 \% で20秒間露光し,出力0 \% で10秒ほど放置する.これを2セット行う.つまり露光時間は合計で2分40秒となる.
\item 現像液をウェハが浸るほどの量をガラスシャーレに入れる.ガラスシャーレを約80秒間,手で揺らし現像する.
\item 現像液からウェハを取り出し,純水でよくすすぐ.その後エアダスタで水気をなくす.この段階でCr電極の先端部分と配線用端子部分にレジストがパターンされている.

\clearpage
\end{enumerate} 

\newpage
\textbf{<h:絶縁層のスパッタリング>}

\begin{enumerate}
\item スパッタ装置を説明書通り立ち上げる. 
\item ウェハの設置はカプトンテープを用いる.ウェハの端をカプトンテープで貼り固定する.
\item スパッタ装置の出力を400 W に設定し14分間SiO$_{2}$をスパッタする.
\item スパッタ装置を説明書通り立ち下げる.この段階でウェハの発泡トリガ面全体にSiO$_{2}$がスパッタされている.
\end{enumerate} 

\vspace{5zh}
\textbf{<i:電極面のリフトオフ>}

\begin{enumerate}
\item アセトンをベムコットンに染み込ませレジストを除去する.この方法をリフトオフ法という.レジストの上にはSiO$_{2}$がスパッタされているためなかなか除去できないがアセトンを染み込ませたベムコットンでよく拭き続ければ除去できる.この段階で発泡トリガの完成である.


\item ヒータ面に貼られている保護テープに紫外線を出力100 \% で5分間照射し,テープを剥がす. MEMS伝熱面の完成である.

\end{enumerate} 