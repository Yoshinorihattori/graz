\section{問題点}
実験結果で確認された単一沸騰気泡がトリガの水素気泡によるものなのか,もしくはMEMS伝熱面下側のヒーターによるものなのか判別ができていない.
すなわち現段階ではトリガ印可時に単一沸騰気泡が任意の発生しているか確認ができていない.


\section{制御装置とタイミング}
テストセクションに配線されているMEMS伝熱面のヒーター加熱時間(待ち時間)と単一沸騰気泡生成装置はArduinoUnoによって制御される.
図\ref{Timing}に時系列と薄膜ヒーター・トリガ・ハイスピードカメラ撮影のタイミングを示した.
ハイスピードカメラの記憶容量の問題から撮影時間は限界がある.
そのためトリガ印可の10$ ms $前からハイスピードカメラの撮影を開始する.
撮影されたハイスピードカメラの気泡画像を確認し,任意のタイミングで気泡が発生しているか確認する必要がある.
制御されたタイミングと気泡画像が一致しない場合はトリガを核としない,ヒーター加熱・ノイズ等によって誤って気泡発生が行われたと考えられるので実験結果として使用できない.
薄膜ヒーター・トリガ・ハイスピードカメラ撮影の3つの電圧はハードディスクロガー(HARD DISK LOGGER GL1000,グラフテック株式会社)によって記録される.

\begin{figure}[ht]
\vspace{1zh}
 \begin{center}
  \includegraphics[width=1\linewidth]{pdf/thesis2017_Timing.pdf}
  \vspace{-1.5zh}
  \caption{薄膜ヒーター・トリガ・ハイスピードカメラ撮影のタイミング}\label{Timing}
 \end{center}
 \vspace{0zh}
\end{figure}

制御装置であるArduinoUnoの最小制御可能時間は1$ ms $であり時間は水晶発振器によって制御されている.
この水晶発振器の精度に疑いがあったのでハードディスクロガー測定間隔200$ us $を用いて制御した時間と実際の時間ではどの程度差があるのかを検討した.
表\ref{Timing}に測定結果を示す.
プログラムによって制御された時間と実際の時間には大きな差がないことがわかる.
これによりArduinoUnoに付属されている水晶発振器で十分に時間制御ができると判断される.

\begin{table}[h]
  \begin{center}
  \caption{Cntrolled and actual time.}\label{プログラムによる制御時間と実際の時間}
  \vspace{0.5zh}
   \includegraphics[width=1.0\linewidth]{pdf/thesis2017_TimingTable.pdf}
   \end{center}
   \vspace{0zh}
\end{table}



\section{制御装置の改良}
単一沸騰気泡は核として水素気泡を電気分解によって発生させる.
井上ら\cite{Inoue2016}による先行研究ではトリガー印可時のノイズが大きく,任意のタイミング以外で水素気泡が発生してしまう可能性があった.
Fig. \ref{test circuit}に薄膜ヒータとトリガの各精密抵抗に流れる電圧を示した.
トリガの抵抗値(純水の抵抗値)はデータロガーの内部抵抗と比較し,大きいため通常の方法で電流を計測することができない.
そこで精密抵抗を直列に配置し,精密抵抗に流れる電流を計測することでトリガに流れる電流を判断する.
本研究ではノイズを少なくするために制御装置の改良を行なった.
具体的には回路に使用していた外部トランジスタを使わずに,フォトカプラ内部のトランジスタを使用した回路を作製した.
テスト回路ではトリガと薄膜ヒータに擬似抵抗を接続し,データロガーで計測することにより実験を行なった.
試験溶液が純水のため実験条件が異なることに注意が必要だが,ノイズを大幅に削減することができた.
%\begin{figure}[ht]
% \begin{center}
%  \includegraphics[width=1.0\linewidth]{figures/interim2017_circuit2016}
%  \vspace{-2zh}
%  \caption{circuit2016}\label{circuit2016}
% \end{center}
% \vspace{-2zh}
%\end{figure}
\vspace{-1zh}
\begin{figure}[h]
 \begin{center}
  \includegraphics[width=1.0\linewidth]{pdf/thesis2017_data_logger_graph_for_paper}
  \vspace{-2zh}
  \caption{Voltage profiles of the trriger and the heater.}\label{test circuit}
 \end{center}
\end{figure}
%\vspace{-2zh}

\newpage

\section{タイミングが判断できないノイズの問題}
試験溶液である純水の抵抗は温度上昇とともに低下することが知られている.
水素イオンと水酸化イオン以外のイオンが全く含まない理論純水の抵抗率は水温 25$°C$で 18.25$MΩ・cm$とされている\cite{Inoue2016}
本研究と同様の電極をNi(ニッケル)で作製した中別府 ら\cite{Nakabeppu2014}の研究では,水の抵抗はおおよそ 100$kΩ$となっている
トリガに電流が流れているか判断するためにはトリガと直列に接続されているシャント抵抗の電流をデータロガーにより計測することにより判断する.
だが,シャント抵抗に流れる電流が微弱でトリガに電流が流れているか判断できないという原因がある.
試験溶液である純水の抵抗は作製したMEMS伝熱面に常温の水滴を垂らすことにより測定し,計測したところおよそ50$MΩ$であった.



\section{異なる熱流束における飽和限界気泡発生時間}
気泡発生装置の加熱液層はMEMS伝熱面下側のヒーター熱流束と加熱時間によって制御される.
ヒーター熱流束もしくは加熱時間どちらかが大きすぎると単一沸騰気泡発生より前に飽和限界を超えてしまい,突沸が起こる.
突沸が起こると過熱液層を破壊してしまい,任意の過熱液層で単一沸騰気泡成長を計測することができない.
過熱液層は突沸が起こらない範囲内で形成する必要がある.
Fig. \ref{LimittedTime}にそれぞれの熱流束における突沸が起こる時間(500$ms$以下)を計測した.
注意点としてヒーター過熱を開始する20$s$前にチャンバーに取り付けられている攪拌器で過熱液層を破壊する.
時間が経つにつれ,チャンバー内の純水が蒸発してしまうので途中で純水を足し,再度40分間の脱気を行なってから実験を行なった.
純水を足した前後で1st, 2ndと区別をした.
この結果からヒーター熱流束,待ち時間の2つのパラメータの設定はFig. \ref{EstimatedHeatingTimeLimit}の左下の領域で実験を行うべきであるとされる.


%\begin{table}[b]
%  \renewcommand{\tablename}{Table. }
%  \caption{Estimated heating time limited.}
%  \label{EstimatedHeatingTimeLimited}
%  \begin{tabular}{|c|p{4cm}|p{4cm}} \hline
%   %\caption{Experimental results at each heat flux.}
%   %\label{Experimental results at each heat flux.}
%   No.&Heat flux[$\rm kW/m^2$]&Heating time[\rm ms]\ \hline
%   1 & 50 & 50\\
%   2 & 100 & 50\\
%   3 & 200 & 50\\ \hline
%\end{tabular}
  
\vspace{-1zh}
\begin{figure}[ht]
 \begin{center}
  \includegraphics[width=1.0\linewidth]{pdf/thesis2017_EstimatedHeatingTimeLimitedForPaper.pdf}
  \vspace{-2zh}
  \caption{Estimated heating time limit.}\label{EstimatedHeatingTimeLimit}
 \end{center}
\end{figure}
 \vspace{-2zh}
