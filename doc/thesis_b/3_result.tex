\section{実験結果}
本実験ではTable.\ref{Experimental results at each heat flux.}に示すように3つの熱流束$q=50\,\rm kW/m^2$, $q=100\,\rm kW/m^2$, $q=200\,\rm kW/m^2$における気泡離脱直径を計測した.
現段階ではトリガ印可時に気泡が発生しているか不明だが,待ち時間は3つの熱流束ともに50$\,\rm ms$である.



Fig. \ref{HeatFlux=50k}に熱流束$q=50\,\rm kW/m^2$, Fig. \ref{HeatFlux=100k}に熱流束$q=100\, \rm kW/m^2$の気泡離脱時の画像を示した.
この結果から熱流束$q=100\,\rm kW/m^2$は熱流束$q=50\,\rm kW/m^2$と比較し気泡離脱直径が大きいことがわかる.
よって過熱液層に伝える熱流束が大きいほど,気泡離脱直径が大きいことがわかる.


つぎにFig. \ref{HeatFlux=200k_with_mark}に熱流束$q=200\, \rm kW/m^2$の気泡離脱時の画像を示した.
単一沸騰気泡ではなく,複数の気泡が発生していることがわかる.
熱流束を大きくすると過熱液層による飽和点を超え,トリガ印可前に気泡が発生することが確認できた.
すなわち熱流束$q=200\, \rm kW/m^2$の条件では待ち時間50$\,\rm ms$では長すぎることがわかった.
%\vspace{-1.0zh}
\begin{table}[b]
  \renewcommand{\tablename}{Table. }
  \caption{Experimental results at each heat flux.}
  \label{Experimental results at each heat flux.}
  \begin{tabular}{|c|p{4cm}|p{4cm}|p{4cm}|} \hline
   %\caption{Experimental results at each heat flux.}
   %\label{Experimental results at each heat flux.}
   No.&Heat flux
   [$\rm kW/m^2$]&Waitting time[\rm ms]&Diameter[\rm mm]\\ \hline
   1 & 50 & 50 & 0.66\\
   2 & 100 & 50 & 1.75\\
   3 & 200 & 50 & No data\\ \hline
   
  \end{tabular}
\end{table}

\begin{figure}[b]
	\begin{minipage}{0.48\linewidth}
    	\includegraphics[width=1\linewidth]{pdf/thesis2017_HeatFlux50k}
        \caption{$q=50\,\rm kW/m^2$ \cite{Inoue2016}}\label{HeatFlux=50k}
	\end{minipage}
	\hfill
	\begin{minipage}{0.48\linewidth}
	     \includegraphics[width=1\linewidth]{pdf/thesis2017_HeatFlux100k}
	     \caption{$q=100\,\rm kW/m^2$}\label{HeatFlux=100k}
	\end{minipage}
	\vspace{-2zh}
\end{figure}

\begin{figure}[h]
 \begin{center}
  \includegraphics[width=0.48\linewidth]{pdf/thesis2017_HeatFlux200k}
  \vspace{0zh}
  \caption{$q=200\,\rm kW/m^2$}\label{HeatFlux=200k_with_mark}
 \end{center}
 \vspace{-2zh}
\end{figure}




%\section{実験結果}
%
%Fig. \ref{9} は熱流束$q=49.4\,\rm kW/m^2$,加熱時間 $\Delta t$ で水素気泡を供給したときの単一沸騰気泡の気泡径を,離脱まで計測した実験結果である.
%$\Delta t$ の範囲は50 ms 〜700 msである.
%印加から発泡までの時間は$0.5\,\rm ms$から$1.5\,\rm ms$であり,バラつきは約$1.5\,\rm ms$以下となった.
%このバラつきは,加熱時間($50\,\rm ms$から$600\,\rm ms$)に成長から離脱までの時間($2.5\,\rm ms$から$14.5\,\rm ms$)を加えた気泡発生の1周期に比べ約30分の1以下と小さく,現象の時間スケールに対して十分な精度で制御できていることが解った.
%
%また,実験結果より,$\Delta t$が大きくなるほど,初期気泡成長速度が速く,気泡離脱径が大きくなることが確認できた.
%
%\begin{figure}[ht]
%\vspace{-0.5zh}
%\begin{center}
%  \includegraphics[width=0.8\linewidth]{gurafu.pdf}
%  \vspace{-0.5zh}
%  \caption{加熱時間 $\Delta$t ,熱流束 $q = 49.4\,\rm kW/m^2$の時の気泡径}\label{9}
% \end{center}
% \vspace{-1.5zh}
%\end{figure}
%
%\begin{figure}[ht]
%\vspace{-0.5zh}
%\begin{center}
%  \includegraphics[width=0.8\linewidth]{souchi.pdf}
%  \vspace{-0.5zh}
%  \caption{実験装置図の概略図}\label{8}
% \end{center}
% \vspace{-0.5zh}
%\end{figure}
