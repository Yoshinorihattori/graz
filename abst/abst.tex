\documentclass[conference]{IEEEtran}
\IEEEoverridecommandlockouts
% The preceding line is only needed to identify funding in the first footnote. If that is unneeded, please comment it out.
\usepackage{cite}
\usepackage{amsmath,amssymb,amsfonts}
\usepackage{algorithmic}
\usepackage{graphicx}
\usepackage{textcomp}
\usepackage{xcolor}
\def\BibTeX{{\rm B\kern-.05em{\sc i\kern-.025em b}\kern-.08em
    T\kern-.1667em\lower.7ex\hbox{E}\kern-.125emX}}
\begin{document}

\title{Experimental investigation of forced convective heat transfer in cylindrical pipe flow\\
%{\footnotesize \textsuperscript{*}Note: Sub-titles are not captured in Xplore and should not be used}
%\thanks{Identify applicable funding agency here. If none, delete this.}
}

\author{\IEEEauthorblockN{Yoshinori Hattori}
\IEEEauthorblockA{\textit{Shibaura Institute of Technology, Mechanical Engineering}\\
Tokyo, Japan \\
md18060@shibaura-it.ac.jp}}

\maketitle

\begin{abstract}
Forced convective heat transfer in cylindrical pipe flow plays an important role in many technical cooling system.
Heat transfer coefficients are vary with flow regime.
Much remains to be study for providing experimental data for transitional regime.
Reliable prediction of heat transfer coefficients for transitional flow is still challenging tasks.
In this study, I focused on forced convective heat transfer in cylindrical pipe flow for transitional regime in particular high Prandtl number.
Moreover, the measurement of wall friction coefficients were also performed in this study.
The engineeres is frequently interested in pressure drop which is related to determine pump or fan power equipments.

\end{abstract}

\begin{IEEEkeywords}
Forced convection, Nusselt number, Wall friction, transitional, Cylindrical pipe flow
\end{IEEEkeywords}

% \section{Introduction}
% In technology, flows regime and heat transfer plays an important role in considerting engineering issues.
% Navie-Stokes equations describe the relation of variable flows.
% However, deterministic solution of the equations are only valid for small disturbances in the initial and boundary condition.
% In physically, it is hard to get initial and boundary conditions in infinite accurate.
% Turbulent has a large amount of fluctuations, i.e. turbulent is completely different kind of laminar flows.\\
% Direct numerical simuration(DNS) is one of the simulation way to predict flow forms.
%
% In recent years, forced convective heat transfer in cylindrical pipe flow plays an important role in many technical cooling systems.
% Nusselt number (Nu) is a dimensionless number which represents the ratio of convective (h) and conductive heat transfer (k), as expressed in Equation \eqref{Nu_number}.
% \begin{equation}
% Nu=\frac{h\cdot L}{k}\label{Nu_number}
% \end{equation}
% From general dimensional analysis, Nusselt number represents function of Reynolds number (Re) times Prandtl number (Pr) as Equation \eqref{Nu_dimensional}.
% \begin{equation}
% Nu=\alpha \cdot Re^{\pi_{\beta}}\cdot Pr^{\pi_{\gamma}}\label{Nu_dimensional}
% \end{equation}
% Here, factors $\alpha$, $\beta$ and $\gamma$ are constant values that depend on flow regime and are calculated from an experimental result.
% Nusselt number is one of the most important numbers for forced convective heat transfer, and are calculated from Equation \eqref{Nu_number} and \eqref{Nu_dimensional}.
%
% Many studies have pointed out that a heat transfer coefficient varies depending on the type of flow: laminar, transition and turbulent.
% Gnienlinski\cite{Gnienlinski2010} showed a calculation method for the laminar heat transfer coefficient of two kinds of boundary conditions. (I) Constant wall temperature (UWT) and (I\hspace{-.1em}I) Constant heat flux (UHF).
% Petukhvov and Kirillov\cite{Petukhov1958} showed calculation method for turbulent flows.
% There has been very scarce experimental data of laminar-to-turbulent transitional region.
% Bertsche et al,\cite{Bertsche2016} focused on reliable prediction of heat transfer coefficient for transitional flows.
% In their study, Bertsche et al, showed experimental heat transfer coefficients for Reynolds number $500 < Re < 23000$ and Prandtl number $7 < Pr < 41$.
%
% Much remains to be studied for providing experimental data except water and glycole as operation fluids.
% In this study, I focused on forced convective heat transfer in flow of water and glycole in a cylindrical pipe.
% A 50/50vol\% mixture of water and glycole, which is a typical liquid coolant in automotive applications was used as a operating fluid.
% The experiment was carried out by considering a board range of Reynolds numbers, spanning from a laminar to fully turbulent flow.
% Moreover, the measurement of wall friction coefficients were also performed in this study.
% The experimental data were compared with other sources as well as computational results obtained from already existing numerical simulations (CFD) by Christphan \cite{Christphan2018}.



\section{Introduction}
Forced convective heat transfer in cylindrical pipe flow plays an important role in many technical cooling systems.
These coolant technology is used wide varaety of coolant applications such as electric devices, automotive, and plant factory.
Considering heat transfer issues, heat transfer coefficients are one of the most important numbers.

Much remains to be studied for providing experimental data for high Pramdlt number and laminar-to-turbulent transitional regime.
In this study, I focus on forced convective heat transfer in cylindrical pipe flow in particular high Pramdlt number and transitional regime.
Shell Heat Transfer Oil was used as a operating liquid.
%A 50/50vol\% mixture of water and glycole which is a typical liquid coolant in automotive applications were used as a operating fluid.


%\section{The phenomena}
%%Temperature variations for constantHeat flux
%\begin{figure}[htbp]
%  \centering
%  %\vspace{-3zh}
%  \includegraphics[width=0.47\textwidth,natwidth=610,natheight=642]{fig/temperature_variations_constant_heat_flux.png}
%  \caption{Temperature variations for constantHeat flux}
%  \label{temperature_variations_constant_heat_flux}
%\end{figure}
%
%%Heat transfer coefficient depend on x value
%Figure\label{heat_transfer_coefficient_depend_on_x_value} shows the local heat transfer coeeficient
%\begin{figure}[htbp]
%  \centering
%  %\vspace{-3zh}
%  \includegraphics[width=0.47\textwidth,natwidth=610,natheight=642]{fig/temperature_variations_constant_heat_flux.png}
%  \caption{Heat transfer coefficient depend on x value}
%  \label{heat_transfer_coefficient_depend_on_x_value}
%\end{figure}
%There are two surface conditions arise in many engineering applications.
%One is constant surface heat flux and the other is constant surface temperature.
%In the fully developed flow of the tube, fluid constant porperties, the local heat coefficient is constant independent of x.


\section{Emperical correlations}%!!!!!!!!!!!!!!!!!!!!!!!!!!!


\subsection{Friction coefficients}
Skin friction coefficient for laminar flow is descrived following equation.
\begin{equation}
    C_{f,lam}=\frac{16}{Re_{b}}
\end{equation}
Konakov\cite{Konakov1954} showed skin friction coefficient for turbulent flow.
\begin{equation}
    C_{f,turb}=0.25(1.8log(Re_{b})-1.5)^{-2}
\end{equation}
Note that these skin friction coefficient just suitable for no-heating condition, constant fluid properties.
In this thesis, we provide heat to the pipe.
Therefore, the fluid properties change depend on the temperature.


\subsection{Heat transfer coefficients}
Gunielinski \cite{Gnienlinski2010} showed correlations for each flow regime, laminar and turbulent, respectively.
Morover, he presented transitional flow regime as a liner interpolation between laminar and turbulent flow.
From general dimensional analysis, Nusselt number represents function of Reynolds number (Re) times Prandtl number (Pr) as following equation.
\begin{equation}
    Nu=\alpha \cdot Re^{\pi_{\beta}}\cdot Pr^{\pi_{\gamma}}\label{Nu_dimensional}
\end{equation}
Here, factors $\alpha$, $\beta$ and $\gamma$ are constant value depend on flow regime and calcurated from numerical experimental results.
Gunienski\cite{Gnienlinski2010} showed correlations for each flow regime laminar and turbulent, respectively.
Gunienski\cite{Gnienlinski2010} showed calculation method for laminar flow.
\begin{equation}
    Nu_{lam}=(3.66^{3}+0.7^{3}+(1.615(Re_{b}Pr_{b}\frac{d_{i}}{L})^{1/3})^{3})^{1/3}\label{Nu_laminar}
\end{equation}
%The range 0.1<<Pr_{b}<<1000, 10^{4}<<Re_{b}<<10^{6}.
Gunienski\cite{Gnienlinski2010} showed calculation method for turbulent flow.
\begin{equation}
    Nu_{turb}=\frac{\frac{C_{f}}{2Re\cdot Pr_{b}}}{1+12.7 \sqrt{\frac{C_{f}}{2}}(Pr_{b}^{2/3}-1)}\cdot (\frac{Pr_{b}}{Pr_{w}})^{0.11}
%Nu_{turb}=\frac{\frac{C_{f}}{2\cdot Re\cdot Pr_{b}}{(1+12.7\frac{C_{f}}{2}}\cdot (Pr^{\frac{2}{3}}-1)}\cdot(1+(\frac{d_{h}}{l})^{\frac{2}{3}})\label{Nu_turblent}
\end{equation}
The range is
\begin{equation}
    0.1<<Pr_{b}<<1000, 10^{4}<<Re_{b}<<10^{6}.
\end{equation}
He presented transitional flow as a liner interpolation between turbulent and laminar flow.
(See ''science problems and interesting issue'' section No.12.)
\begin{equation}
    Nu_{m}=(1-r)Nu_{m,lam}+rNu_{m,turb}
    \label{Nu_m}
\end{equation}
\begin{equation}
    r=\frac{Re_{b}-2300}{10^{4}-2300}
\end{equation}


\section{Experimental setup}
%Experimental Setup
Figure. \ref{experimental_loop} shows experimental loop.
The experimental loop consists of heat exchanger, pump, coriolis mass flow rate, welder, reservoir, and test section basically.
Heat exchenger keep thermal stationary condition in flow pipe.
Mass flow rate is controlled by pump and baypass valve C which is located in a parallel.
The pipe is thermal isolated, surrounded with glass wool.\\

\begin{figure}[htbp]
\centering
\vspace{-4zh}
\includegraphics[width=0.47\textwidth,natwidth=920,natheight=700]{fig/experimental_loop.png}
\caption{Process flow diagram of the test facilities including test section.}
\label{experimental_loop}
\end{figure}

Figure. \ref{thermal_boundary_layer_development} shows velocity and thermal boundary layer development vary with horizontal axis in a test section.
Velocity and thermal profile are shown blue and red color, respectively.
The test section is made of stainless steel (1.4301) with an inner diameter di=12mm and outer diameter do= 15mm.
Highly accurate resistance thermall probes (PT-100) are used to find out the inlet and outlet bulk temperature (Tib , Tob) and wall temperature Tw.
Moreover, thermocouple 'Type-K' are used to take temperature gradient in flow direction.

The test section consist of entrance, heated and thermal equalized part.
\begin{enumerate}
  \item Entrance part\\
  The first part of test section is 1.2 [m] length entrance part which is sufficiently long to ensure dynamically developed flow condition at the exist.
  The bulk temperature (Tb0) at this section were measured by PT-100.
  \item Heated part\\
  The second part of test section is 2 [m] length heated part which is sufficiently long to ensure thermaly fully developed flow condition at the exist.
  The tube wall were heated electrically by welder which provide high current and low voltage to keep the uniform heat flux condition in a inner pipe flow.
  Convective heat transfer is independent with horizontal axis in fully developed flow, constant heat flux condition.
  The wall temperature (Tw) at the exist of this section were measured by PT-100.
  \item Thermal equalized part\\
  The third part of test section is thermall equalized part which is including static mixture. Static mixture forms turbulent and vortex. Then, the thermal profile of heated exist mix together. At the end, the bulk temperature(Tb1) are measured.
\end{enumerate}
\begin{figure}[htbp]
  \centering
  %\vspace{-3zh}
\includegraphics[width=0.47\textwidth,natwidth=850,natheight=450]{fig/thermal_boundary_layer_development.png}
  \caption{Velocity (blue) and thermal (red) boundary layer development vary with horizontal axis in a test section.}
  \label{thermal_boundary_layer_development}
\end{figure}

\newpage

\subsection{Length-to-diameter ratio}
%Ref. Experiments in Pipe Flows at Transitional and Very High Reynolds Numbers p25-
The length-to-diameter ratio is an important parameter to achieve the fully developed turbulent condition in the test section.
Test section has an inner diameter of $D=12mm$, and the length of $L=2m$ which length-to-diameter ratio is $L/D=167$.
Patel et.al.\cite{Patel1969} showed suitable the length-to-diameter ratio for fully developed turbulent flows.
According to their study, they found that the minimum developing length of $L/D=70D$.
Therefore, the length-to-diameter ratio of the test section in this experimental is long enough to ensure a fully developed turbulent flow state.

\subsection{Conduction equation}
TPT100 sensors are attached
\subsection{Wall roughness}
%Wall properties
To clarify the wall surface as a smooth pipe, wall roughness was considered.
Moody diagram defined the basis of friction chart, and that can be used in practice.
Nikuradse made a throughout studies of turbulent flows in pipes with a rough surface.
Reynold number can be interpreted as the ratio between the rough height $h$ and thickness of the viscous sublayer $\nu/u_{*}$, as the following equation.
\begin{equation}
    Re=\frac{hu_{*}}{\nu}
\end{equation}
Here, $u_{*}$ represents wall friction velocity and descrived following equation.
\begin{equation}
    u_{*}=\sqrt{\frac{\tau_{wall}}{\rho_{0}}}
\end{equation}
Wall roughness of a characteristic height is descrived following equation.
\begin{equation}
    h = 4k_{rms}
\end{equation}
In this experiment, the surface roughness of interior surfaces in stainless steel (1.4301) is approximately $k_{rms} = 5\mu m$.\\
Table\ref{wall_roughness} shows roughness consideration in this experiment.
%at maximim Re number
\begin{table}[h]
    \caption{Oders of each terms in Equation}
    \label{wall_roughness}
    \centering
    \begin{tabular}{ccccc}
        \hline
        surface roughness & roughness height & viscous sublayer & Re \\
        \hline
        5$\mu m$ & 20$\mu m$ & 0.4$\mu m$ & 0.48
    \end{tabular}
\end{table}
It is shown that the Reynolds number in this experiment is less than 1.
Thus, the interior wall surface in the experimental setup is considered to be a smooth surface.

\subsection{Velocity profiles}
Turbulent velocity profiles for the smooth wall as following equation.
\begin{equation}
    \overline{u} = u_{*} \left[\frac{1}{k}ln\left(\frac{y}{h}\right)+B^{\prime}\right]
\end{equation}
$B^{\prime}$ is a function of the Reynolds number and calcurated as following equation.
\begin{equation}
    B^{\prime} = 2.5 ln\left(\frac{hu_{*}}{\nu}\right)+5
    %for :\frac{hu_{*}}{\nu}<<1 (i.e. a smooth wall)
\end{equation}

Figure\ref{mean_velocity_profile} shows mean velocity profile....
\begin{figure}[htbp]
  \centering
  \vspace{5zh}
  \includegraphics[width=0.47\textwidth,natwidth=400,natheight=200]{fig/velocity_profile.pdf}
  \vspace{-1.5zh}
  \caption{mean velocity profile when Re=2300 and Pr=50.}
  \label{mean_velocity_profile}
\end{figure}

\section{Effective data procedure}


\section{Calucuration Flow}
Material properties are all temperature-dependent function.
At first, material properties varies with temperature were taken.
Next, I move to experimental facilities and measured temperature differences, pressure differences and mass flow rates.
Finaly, Nusselt, Prandtl, Reynolds numbers and friction coefficients were calcurated by post-proccesing, LabView and MATLAB.
\begin{figure}[htbp]
  \centering
  \vspace{5zh}
  \includegraphics[width=0.47\textwidth,natwidth=400,natheight=200]{fig/post_processing.pdf}
  \caption{Post processing}
  \label{post_processing}
\end{figure}

\section{Results and Discussions}
The aim of this research is to investigate heat transfer and friction coefficients using  Water glycole $50\%/50\%$ mixture as an oparatiog liquid.
In order to investigate Prandtl number throughly, first the experimental results for Pr=50 is compared with data obtained in previous studies \cite{Christphan2018} using Shell Heat Transfer Oil for Prandtl numbers Pr=50.
Therefore, the validity of evaluation process for the experiment are checked.


\subsection{Heat transfer coefficients for $Pr=50$}
Figure. \ref{renu_pr50} shows dimensionless heat transfer coefficients compared to emperical correlations Eqs() and previous studies.
\begin{figure}[htbp]
  \centering
  \vspace{5zh}
  \includegraphics[width=0.47\textwidth,natwidth=400,natheight=200]{fig/renu_pr50.pdf}
  \vspace{-1.5zh}
  \caption{Dimensionless heat transfer coefficients compared to literature data and previous studies for Pr = 50. The red, yellow and blue lines are Gnielinski correlations for laminar, transitional and turbulent, respectively.}
  \label{renu_pr50}
\end{figure}

It is nessesary to estimate the maximum possible error in the parameters evaluated from the measuring data.
Nusselt number is calcuretad from Equation \ref{Nu_calcuration}.
\begin{equation}
    Nu=\frac{\dot{m} c_{p}(T_{1}-T_{0})}{\lambda (T_{w}-T_{1}) d\pi L}\label{Nu_calcuration}
\end{equation}
The uncertainty in Nusselt number is calcurated from Equation \ref{Nu_uncertainty1}.
\begin{equation}
    Nu=\sqrt{\sum \left(\frac{\partial Nu}{\partial X_{i}} \Delta X_{i}\right)^{2}}\label{Nu_uncertainty1}
\end{equation}
Each parameter, X effects each parameters of Equation.\ref{Nu_calcuration}
Then, the mesurement uncertainty in Nusselt number leads to Equation \ref{Nu_uncertainty2}.
\begin{equation}
    \begin{split}
        &\frac{\Delta Nu}{Nu} =\frac{1}{Nu}\sqrt{\left(\frac{\Delta \dot{m}}{\dot{m}}\right)^{2}+\left(\frac{\Delta c_{p}}{c_{p}}\right)^{2}+\left(\frac{\Delta \lambda}{\lambda}\right)^{2}}\\
        &\quad +\left(\frac{\Delta T}{T_{0}-T_{1}}\right)^{2}+\left(\frac{\Delta T(T_{0}-T_{w})}{(T_{0}-T_{1})(T_{1}-T_{w})}\right)^{2}+\left(\frac{\Delta T}{T_{1}-T_{w}}\right)^{2}\\\label{Nu_uncertainty2}
    \end{split}
\end{equation}
%I estimated mesurement error of mass flow and temperature sensors are 0.20E-3$\dot{m}$, 0.2K.
%Specific heat capacity and heat conductivity are calcurated from fluid properties, temperature dependance.
%Specific heat capacity and heat conductivity are function of temperature as shown in Equation \ref{specific_heat_capacity} and \ref{heat_conductivity}.
%\begin{equation}
%C_{p} = 818 + 3664T \times 10^{-3} \label{specific_heat_capacity}
%\end{equation}
%\begin{equation}
%\lambda = 0.157 - 7.328T \times 10^{-5} \label{heat_conductivity}
%\end{equation}
%Therefore, mesurement error of specific heat capacity Cp and heat conductivity $\lambda$ are 0.73 and 1.56E-5, respectively.
%Here, I picked up two experimental data.
%Shell Heat Transfer Oil was used as a operating liquid.
%Table\ref{experimental_result} shows experimental results for Re=4275, Pr=26 and Re=6172, Pr=27, respectively.
%
%\begin{table}[h]
% \caption{Esperimental results for Re=4275, Pr=26 and Re=6172, Pr=27.}
% \label{experimental_result}
% \centering
%\begin{tabular}{llllllll}
%\hline
%       & Re   & Pr & T0    & T1    & Tw     & Massflow & Nu   \\ \hline
%CASE A & 4275 & 26 & 438.25 & 446.85 & 472.45 & 0.054    & 56.0 \\
%CASE B & 6172 & 25 & 452.87 & 452.87 & 478.97 & 0.073    & 81.6
%\end{tabular}
%\end{table}
%The resulting uncertainty in nusselt numbers were analyzed for each of 2 data points.
%Table\ref{uncertainty_in_Nusselt_number} summrizes uncertainty in Nusselt number of each data points.
%
%\begin{table}[h]
% \caption{Uncertainty in Nusselt number.}
% \label{uncertainty_in_Nusselt_number}
% \centering
%\begin{tabular}{clc}
%\hline
%\multicolumn{3}{l}{Uncertainty in Nusselt number}         \\ \hline
%CASE   & \multicolumn{1}{c}{Flow regime, Pr} & Nusselt number \\ \hline
%CASE A & Re = 4275, Pr = 26              & 4\%            \\
%CASE B & Re = 6172, Pr = 25              & 3\%
%\end{tabular}
%\end{table}
%Figure. \ref{uncertainty} shows heat transfer coefficients compared to literature data for transitional regime, Pr = 26.
%Two data shows good agreement with calcuration method for transitional regime proposed by Gunielinski \cite{Gnienlinski2010}.


\subsection{Friction coefficients for $Pr=50$}
%Friction coefficient
Figure. \ref{recf_pr50} shows friction coefficients compared to emperical correlations Eqs() and previous studies.
\begin{figure}[htbp]
  \centering
  \vspace{5zh}
  \includegraphics[width=0.47\textwidth,natwidth=400,natheight=200]{fig/recf_pr50.pdf}
  \caption{Friction coefficients compared to literature data for Pr = 50. The red and blue lines are emperical correlations for laminar and turbulent flows, respectively.}
  \label{recf_pr50}
\end{figure}
It is obvious that the experimental results on friction coefficient for $2200 < Re < 7000$ are very well fitted by the citted correlation.


\subsection{Purpose of the solution to deduce mesurement uncertainty for Nu}
Table \ref{order} shows the orders of each terms in Equation \ref{Nu_uncertainty2}.
From this comprehensive result, temperature term is several orders of magnitude larger than others.
Therefore, temperature mesurement is strongly inflected to mesurement uncertainty.
%for the high-order terms are considerd to be less important than those of low order terms.
According to this analisis, careful selections of temperature sensors are needed.
\begin{table}[h]
 \caption{Oders of each terms in Equation \ref{Nu_uncertainty2}.}
 \label{order}
 \centering
 \begin{tabular}{ccccc}
\hline
      & Mass flow & Specific heat capacity & lambda & Temperature \\ \hline
Order & O (-8)    & O (-8)                 & O(-8)  & O(-4)
 \end{tabular}
\end{table}

\section{Future plan}


%\section{Measurement uncertainty}
%The uncertainty of measurement was analyzed by using ``Guide to the Expression of Uncertainty in Measurement``(GUM).
%\\
\newpage
\section{Appendix}
The experiment was carried out already existing fasilities by Christphan2018 \cite{Christphan2018}.
Experimental procedure is as follows.
\begin{enumerate}
  \item Switch on (Ein) Main switch (S0)
  \item Start up a computer
  \begin{enumerate}
      \item Select ``Rohro.lvproj''
      \item Select Lab VIEW and click ``Starten''
      \item Click ``Nein''
      \item Select ``Rohrpufsp.vi''
  \end{enumerate}
  \item Prepare water supply for cooling experimental facilities.
  \begin{enumerate}
      \item Open the tap water and save cool water in big tank.
      \item Cehck the temperature is approximately 15$^\circ$C.
      \item Cehck the valves are in following state.
      Valve A is closed, B is opened, and C is closed.
      \item Turn on pump swith to supply cooling water
  \end{enumerate}
  \item Check the value of mass flow rate on PC display and wait until the value is approximately 0.
  \item Click $T_{s}$ allec and $T_{s}$ gleci to carry out Temperature calibration
  \item Switch on heater (S1: Heizstab)
  \item Click ``Aushmine'' to start the experiment
  \item Set ``Welder stom'' under 300A
  \item Switch on the pump (S4: Pump)
  \item Switch on the welder
  \item Adjest suitable experimental parameters for Re and Pr.
  \begin{enumerate}
      \item Decrease Re and increase Pr\\
      Open valve A and close valve B little by little.
      Enhance cooling, statics viscosity increase.
      Thus, Pr increase.
      \item Decrease Re and decrease Pr\\

      \item Decrease Re and keeps Pr constant\\
      
  \end{enumerate}
  \item Wait until the condition reaches steady
  \item Click ``Schreiben'' to save the data
  \item Switch of the Welder
  \item Open the valves (flow max) and cool down the facilities
  \item Wait until the facilities cool down enough
  \item Turn off pump swith to stop supplying cooling water
  \item Switch off the pump (S4: Pump)
  \item Close the tap water
  \item Click ``close'' for LabView application
  \item Shut down PC and click ``Klick sie heir unclear----'', not to download any current version
\end{enumerate}

\begin{thebibliography}{00}
%\bibitem{Frank} Frank P. Incropera, David P. DeWitt, ``Fundamentals of Heat and Mass Transfer,'' 4th edition., WILEY, 1996.
\bibitem{Patel1969}Patel and Head
\bibitem{Konakov1954}Konakov
\bibitem{Gnienlinski2010} V. Gnienlinski, ``Heat Transfer in Laminar Flow,'' VDI Heat Atlas, second ed., Springer Verlag, 2010 (Chapter Ga 1-7), Section 3.
%\bibitem{Petukhov1958} B.S. Petukhov, V.V. kirillov, Teploenergetika 4 (1958) 91-98.
\bibitem{Bertsche2016} Dirk Bertsche, Paul Knipper, Thomas Wetzel, ``Experimental investigation on heat transfer in laminar, transitional and turbulent circular pipe flow,'' International Journal of Heat Transfer, 95 (2016) 1008-1018.
\bibitem{Christphan2018} Christphan, ``Title of paper if known,'' unpublished, 2018.
\bibitem{Emir2018}Emir Öngüner, 'Experiments in Pipe Flows at Transitional and Very High Reynolds Numbers', Cuvillier Verlag, 2018.
\bibitem{Petukhov1958}Petukhov


%bibitem{Konakov1954} Konakov, ``Eine neue Formel fr den Reibungskoezienten glatter Rohre,'' Bericht der Akademie der Wissenschaften der UDSSR 51.7, 503-506.

% \bibitem{b5} R. Nicole, ``Title of paper with only first word capitalized,'' J. Name Stand. Abbrev., in press.
% \bibitem{b6} Y. Yorozu, M. Hirano, K. Oka, and Y. Tagawa, ``Electron spectroscopy studies on magneto-optical media and plastic substrate interface,'' IEEE Transl. J. Magn. Japan, vol. 2, pp. 740--741, August 1987 [Digests 9th Annual Conf. Magnetics Japan, p. 301, 1982].
% \bibitem{b7} M. Young, The Technical Writer's Handbook. Mill Valley, CA: University Science, 1989.

%\bibliography{junsrt}
%\bibliography{project_forced_convective@Graz.bib}

\end{thebibliography}
%\vspace{12pt}
%\color{red}
%IEEE conference templates contain guidance text for composing and formatting conference papers. Please ensure that all template text is removed from your conference paper prior to submission to the conference. Failure to remove the template text from your paper may result in your paper not being published.

\end{document}
